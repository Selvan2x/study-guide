\section{Studie Materiaal en Sagteware}
    \subsection{Voorgeskrewe Handboek}
        Daar is geen voorgeskrewe handboek vir hierdie kursus nie. Die volgende
        studienotas is by UP ontwikkel en sal beskikbaar wees op
        \textit{ClickUP}:

        ``\underline{Introduction to programming for engineers using Python}''
        deur Logan G. Page, Daniel N. Wilke en Schalk Kok.

    \subsection{Komplement\^{e}re Bronne}
        Die volgende aanvullende bron van notas sal ook beskikbaar gestel word
        op \textit{ClickUP}:

        ``\underline{Python for Computational Science and Engineering}'' deur
        Hans Fangohr

        Talle aanvulllende bronne vir hierdie kursus word bespreek in die
        bogenoemde studienotas.  Die besprekingsforum en wiki-bladsy op die
        \textit{ClickUP} stelsel sal ook gebruik word vir addisionele
        verduidelikings en voorbeelde oor verskeie kursusonderwerpe.

        Notas van addisionele studiemateriaal (wat nie gedek word in die
        voorgeskrewe notas nie) sal elektronies beskikbaar wees op
        \textit{ClickUP}. Hierdie notas sal ook deel wees van die sillabus.
        Let wel, hierdie addisionele notas dek nie \'{a}l die werk wat bespreek
        word gedurende klas nie en studente moet asseblief hulle eie
        aanvulllende notas maak gedurende klastye.

        Uitgewerkte oplossings wat gedurende die lesings in detail gedek word,
        sal nie later beskikbaar gemaak word op \textit{ClickUP} nie.

    \subsection{Vereiste Sagteware}
        In hierdie kursus maak ons gebruik van vrylik beskikbare sagteware.
        Dit impliseer dat studente die sagteware kan gebruik en kopie\"er
        sonder enige wettige beperkings.

        Die programmeringspakket wat gebruik word in hierdie kursus is Python.
        Python is 'n ho\"{e}-vlak programmeringstaal wat gebruik word in baie
        dissiplines reg oor die w\^{e}reld. Python is 'n goed geskikte
        programmeringstaal vir ingenieurs omdat dit maklik is om te leer.
        Python is ook goed gedokumenteer, relatief vinnig en goed geskik vir
        numeriese en ingenieurs berekeninge.

        Jy kan Python aflaai ({\tt Python(x,y)-2.7.9.0.exe}) by:
        \begin{itemize}
            \item \textit{ClickUP} (aanbeveel) of
            \item \url{http://code.google.com/p/pythonxy/wiki/Downloads}
        \end{itemize}
        (Sien die bogenoemde studienotas vir installeeringsinstruksies)

        In hierdie kursus word LibreOffice Calc gebruik vir sigblaaie
        (spreadsheets).  LibreOffice Calc is grootendeels soortgelyk aan
        Microsoft Excel, met die uitsondering dat dit afgelaai en gebruik kan
        word sonder wettige beperkings.

        Jy kan LibreOffice aflaai:
        ({\tt  LibreOffice\_4.3.5\_Win\_x86.msi}) van:
        \begin{itemize}
            \item \textit{ClickUP} (aanbeveel) of
            \item \url{http://www.libreoffice.org/download/libreoffice-fresh/?type=win-x86&version=4.3&lang=en-ZA}
        \end{itemize}
        (Sien \url{http://www.libreoffice.org/get-help/install-howto/windows/}
        vir installeringsinstruksies)

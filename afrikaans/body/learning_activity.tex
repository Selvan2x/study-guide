\section{Leer Aktiwiteite}
    \subsection{Kontaktyd en Leerure}
        \begin{minipage}{0.4\linewidth}
            Aantal lesings per week: \\
            Tutoriaalsessies per week:
        \end{minipage}
        \begin{minipage}{0.4\linewidth}
            4 lesings, 50 minute per lesing \\
            1 sessie, 100 minute per sessie
        \end{minipage}

        Hierdie module dra `n gewig van 16 krediete, wat aandui dat `n student
        gemiddeld 160 ure moet bestee om die vereiste vaardighede te bemeester
        (insluitend die voorbereidingstyd vir toetse en eksamens). Dit beteken
        dat die student gemiddeld 11 ure per week se studietyd moet afstaan aan
        hierdie module. Die geskeduleerde kontaktyd vir hierdie module is
        ongeveer 6 ure, wat beteken dat nog 6 ure aan selfstudie toegewy moet
        word.

    \subsection{Lesings}
        \begin{table}[!h]
            \begin{center}
             \begin{tabular}{|l|c|p{1.4cm}|p{2.6cm}|p{2.1cm}|p{2.6cm}|p{1.4cm}|}
                 \hline
                 {\bf \#} & {\bf Tyd} & {\bf Ma.} & {\bf Di.} & {\bf Wo.} &
                 {\bf Do.} & {\bf Vry.} \\
                 \hline
                 1  & 07:30--08:20 &  & [E2]Eeufees Centenary 4 & [A]Ing III--6 &  & \\ \hline
                 2  & 08:30--09:20 &  & [E2]Eeufees Centenary 4 & [A]Ing III--6 &  & \\ \hline
                 3  & 09:30--10:20 &  &  &  &  & \\ \hline
                 4  & 10:30--11:20 &  &  &  &  & \\ \hline
                 5  & 11:30--12:20 &  &  &  & [E1]Thuto 1--1 & \\ \hline
                 6  & 12:30--13:20 &  &  &  & [E1]Thuto 1--1 & \\ \hline
                 7  & 13:30--14:20 &  &  &  &  & \\ \hline
                 8  & 14:30--15:20 &  &  &  &  & \\ \hline
                 9  & 15:30--16:20 &  & [E1]Thuto 1--2 &  & [E2] Eng III--1 & \\ \hline
                 10 & 16:30--17:20 &  & [E1]Thuto 1--2 [A]Ing III--6 &  & [E2] Eng III--1 & \\ \hline
                 11 & 17:30--18:20 &  & [A]Ing III--6 &  &  & \\
                 \hline
             \end{tabular}
             \caption{Lesingrooster}
            \label{tab:lectures}
            \end{center}
        \end{table}

        Mnr Roux sal die Afrikaanse lesings aanbied en Mnr Page sal die Engelse
        lesings aanbied soos aangedui in Tabel~\ref{tab:lectures}. Die lesings
        word verdeel soos aangedui op die rooster. Verwys asseblief na jou
        individuele rooster om te weet watter Engelse lesing groep jy moet
        bywoon. Verwys ook na \textit{ClickUP} vir enige veranderinge in
        verband met die rooster.

        Lesing bywoning en deelname aan besprekings is verpligtend.  Aangesien
        die inhoud van elke lesing volg op di\'{e} van die vorige lesings, is
        dit in die student se eie belang om die materiaal te bestudeer en nie
        `n lesing te mis nie. As `n student `n lesing nie kan bywoon nie, is
        dit die verantwoordelikheid van die student om die studiemateriaal te
        kry en die werk in te haal.  Geen individuele lesings sal aangebied
        word nie. Materiaal sal selde eksplisiet in lesings herhaal word.

        Studente word aangeraai om voor te berei vir die lesings. Die
        student word dan vir `n tweede keer aan die materiaal tydens
        die lesings blootgestel. Dit sal die student ook die
        geleentheid gee om op probleem areas te fokus tydens die
        lesings. Aangesien `n groot volume werk gedek moet word, is
        dit nie moontlik om elke aspek van die werk in besonderheid te
        bespreek nie.

        Die dosent mag ook sekere dele van die studienotas as selfstudie aandui
        of materiaal vir selfstudie uitdeel. Hierdie inhoud sal deel vorm van
        die vakinhoud, maar nie eksplisiet tydens lesings bespreek word nie,
        tensy `n student dit vir bespreking tydens `n lesing opper.

    \subsection{Tutoriale}
        \begin{table}[!h]
            \begin{center}
            \begin{tabular}{|l|l|l|l|}
                \hline
                {\bf Dissipline} & {\bf Dag} & {\bf Tyd} & {\bf Lokaal} \\
                \hline
                S3          & Mo.  & 11:30-13:20 & NWII Lab 1,2,3,4 \\
                c3 s3 C2    & Mo.  & 15:30-17:20 & NWII Lab 2,3,4 \\
                M2          & Di.  & 07:30-09:20 & NWII Lab 2,3,4 \\
                n3 p3 N2 P2 & Wo.  & 11:30-13:20 & NWII Lab 1,2,3,4 \\
                b3 B2       & Do.  & 15:30-17:20 & NWII Lab 1,2,3,4 \\
                m3 M2       & Vr.  & 13:30-15:20 & NWII Lab 1,2 \\
                \hline
            \end{tabular}
            \caption{Tutoriaalrooster}
            \label{tab:tutorials}
            \end{center}
        \end{table}

        Tutoriaal sessies sal gebruik word vir verdere ontwikkeling van die
        studente se begrip en kennis oor die lesingsonderwerpe.  Die
        tutoriaalsessies dien as hulpmiddel om die student in staat te stel om
        verskeie ingenieurs- en wiskundigeprobleme op te los d.m.v. gereedskap
        ontwikkel in die kursus. Die tutoriaalsessies word getabuleer in Tabel
        \ref{tab:tutorials}, en is toegeken volgens
        dissipline\footnote{Dissipline Simbole:
            B -- Bedryfs en Stelsels;
            C -- Chemies;
            M -- Meganies en Lugvaartkundige;
            N -- Materiaalwetenskap en Metallurgies;
            P -- Mynbou; and S - Siviel \\
            (hoofletter -- vierjaarplan; kleinletter -- vyfjaarplan)}.

        Elke week se tutoriaal sessie sal gebruik word om die vorige week se
        lesing inhoud en konsepte te hersien . Gekose oefenprobleme uit die
        studie notas, sal tydens die tutoriaalsessies gedek word. Die gekose
        hoofstukke en probleme wat tydens elke tutoriaalsessie gedek word sal
        op {\it ClickUP} aangekondig word.

        Dit is noodsaaklik dat studente hierdie tutoriaalprobleme op hul eie
        deurwerk, voor die bywoning van die tutoriaalsessie. Studente word
        sterk aangeraai om die tutoriaalsessies te gebruik om aspekte van
        probleemoplossing en / of konsepte waarmee die student sukkel aan te
        spreek. Dit is belangrik om te verstaan dat die tutors daar is om die
        student te help om moeilike konsepte te begryp wat die student hinder
        om `n probleem op te los. Die tutors is nie daar om die die probleme
        namens die studente op te los nie.

        Dit sal tot die voordeel van die student wees om hierdie probleme te
        voltooi, aangesien die oplos van hierdie probleme die student se denke
        en probleem oplos vaardigheid ontwikkel.  Die toetse en eksamen
        asseseer of die student nuwe probleme met soortgelyke kompleksiteit kan
        oplos.

        \textbf{Bywoning} van die \textbf{tutoriaalsessies} is
        \textbf{opsioneel} maar die \textbf{inhanding} van die
        \textbf{tutoriaalopdragte} op {\it ClickUP} is \textbf{verpligtend},
        aangesien hulle \textbf{gemerk} gaan word en \textbf{tel tot die
        semesterpunt}.

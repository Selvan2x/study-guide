\section{Departementele Studiegids}\label{sec:department}
    Die studiegids is `n belangrike deel van die algemene studiegids van die
    Departement. In die Departementele studiegids word informasie gegee oor die
    visie en missie van die departement, algemene administrasie en regulasies
    (professionaliteit en integriteit, kursus verwante inligting en formele
    kommunikasie, werkswinkel gebruik en veiligheid, plagiaat,
    klasverteenwoordiger pligte, sieketoets en siek eksamenriglyne,
    vakansiewerk, app\'{e}lproses en aanpassing van punte,
    universiteitsregulasies, vrae wat dikwels gevra word, ens.), ECSA uitkomste
    en ECSA uittreevlak uitkomste, ECSA kennisarea, CDIO, nuwe kurrikulum, en
    assessering van kognitiewe vlakke.  Daar word verwag dat jy baie vetroud
    word met die inhoud van die Departementele Studiegids. Dit is beskikbaar in
    \href{http://www.up.ac.za/media/shared/120/Noticeboard/2017/departmental-studyguide-eng-2017.zp107056.pdf}{Engels}
    en
    \href{http://www.up.ac.za/media/shared/120/Noticeboard/2017/departementele-studiegids-afr-2017.zp107058.pdf}{Afrikaans}
    op die
    \href{http://www.up.ac.za/en/mechanical-and-aeronautical-engineering/article/21692/noticeboard}{Departementele webwerf}.

    \noindent
    \textbf{Engels:} \\
    \url{http://www.up.ac.za/media/shared/120/Noticeboard/2017/departmental-studyguide-eng-2017.zp107056.pdf} \\~\\
    \textbf{Afrikaans:} \\
    \url{http://www.up.ac.za/media/shared/120/Noticeboard/2017/departementele-studiegids-afr-2017.zp107058.pdf} \\~\\
    \textbf{Departementele Webwerf:} \\
    \url{http://www.up.ac.za/en/mechanical-and-aeronautical-engineering/article/21692/noticeboard} \\~\\

    \noindent
    \textbf{Neem kennis van die \uline{spesifieke instruksies} soos uiteengesit
    in die studiegids hierbo:}
    \begin{itemize}
        \item \textbf{Veiligheid}
        \item \textbf{Plagiaat}
        \item \textbf{Wat om te doen indien jy siek was (baie belangrik)?}
        \item \textbf{App\'el proses vir punte aanpassings}
    \end{itemize}

\newpage
\section{Erkenning van Modules} \label{sec:credit_study_guide}
    \noindent
    Krediet-erkenning vir MPR213, gebaseer op erkenning van ander modules,
    sal slegs toegestaan word indien die student
    \begin{enumerate}
        \item voldoen aan die stipulasies uiteengesit in die algemene
            regulasies en re\"els van die Universiteit (sien
            \url{http://www.up.ac.za/af/yearbooks/rules}),
        \item `n uittreevlak uikomste gebaseerde krediet assesserings eksamen
            slaag (50\%+),
        \item nog nooit die krediet assesserings eksamen in die verlede (voor
            2017) geskryf het nie.
    \end{enumerate}

    \noindent
    Toelating tot die krediet assesserings eksamen vereis dat die student
    \begin{enumerate}
        \item huidiglik geregistreer is vir MPR213,
        \item die krediet aansoekvorm voltooi, wat by studente administrasie
            (Eng I, floor 6) beskikbaar is,
        \item die voltooide kredietvorm saam met `n volledige akademiese rekord
            en studiehandleidings van die reeds geslaagde modules inhandig by
            Mnr Page (Eng III 6-93). Die laaste datum van inhandiging is om
            17h00 op 17 Februarie 2017.
    \end{enumerate}

    \noindent
    Inligting omtrent die krediet assesserings eksamen:
    \begin{itemize}
        \item Datum: 24 Februarie 2017
        \item Tyd: 14:00 to 15:30 (90min)
        \item Lokaal : NW2 PhD Lab
        \item Omvang:
        \begin{itemize}
            \item 30\% LibreOffice (sigblaaie) -- Meervuldige keusevrae
            \item 70\% Python (probleemoplossing) -- Geskryfde formaat
            \item Additionele inligting is beskikbaar in die klasnotas op
                ClickUP.
        \end{itemize}
        \item Jy sal toegang tot `n rekenaar met Python en LibreOffice h\^e
            tydens die eksamen om te help om die probleme mee op te los.
        \item Al die oplossings moet in `n antwoordboek geskryf word.
        \item Slegs die antwoordboek word ingehandig, geen kode sal elektronies
            gestoor word nie.
        \item Daar is geen additionele ``siekte'' eksamen nie. Indien hierdie
            eksamen verpas word sal die student die module moet voltooi.
        \item Enige veranderinge sal per e-pos / ClickUP gekommunikeer word.
    \end{itemize}


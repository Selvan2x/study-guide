\section{Study Material and Software}
    \subsection{Prescribed Textbook}
        There is no prescribed textbook for this course, however the following
        \textbf{study notes} have been developed at UP and will be made
        available on {\it ClickUP}:

        ``\underline{Introduction to programming for engineers using Python}''
        by Logan G. Page, Daniel N. Wilke, and Schalk Kok.

    \subsection{Complementary Sources}
        The following complementary source of notes will also be made available
        on {\it ClickUP}:

        ``\underline{Python for Computational Science and Engineering}'' by
        Hans Fangohr

        Numerous other complementary sources for this course are mentioned in
        the study notes above. The discussion forum and wiki page on the {\it
        ClickUP} system will also be utilised for additional explanations and
        examples on various topics covered in this course.

    \subsection{Lecture Notes}
        Take note, we make the distinction here between \textbf{study notes}
        discussed above and \textbf{lecture notes} discussed here. These
        lecture notes will also form part of the syllabus and are available on
        \textit{GitHub}:

        \url{https://github.com/mpr213/lecture-notes/releases}

        These lecture notes do not cover all the work discussed in class and
        students should work through the study notes and take down their own
        supplementary notes during lectures. Problem solutions covered in
        detail during the lectures will not be made available again at a later
        stage.

    \subsection{Required Software}
        In this course, we make use of open source software. This implies
        that students can freely copy and use the software legally
        without any restrictions.

        The programming package used in this course is Python. Python is
        high level programming language used in many disciplines around the
        world. Python is a well suited programming language for engineers as
        it is easy to learn, well supported and documented, relatively fast,
        and well suited for numerical and scientific computing.

        LibreOffice Calc will be used for spreadsheets in this course. It is
        in many ways very similar to Microsoft Excel with the exception
        that it can be downloaded and used without any restrictions.

        \noindent
        You can download Python via Anaconda3 from:
        \begin{itemize}
            \item Online: \url{https://www.continuum.io/downloads}
            \item Campus FTP (Windows): \url{ftp://ftp.ee.up.ac.za/pub/windows/python}
            \item Campus FTP (Mac OSX): \url{ftp://ftp.ee.up.ac.za/pub/mac/python}
            \item Campus FTP (Linux): \url{ftp://ftp.ee.up.ac.za/pub/linux/python}
        \end{itemize}
        \textbf{Note:} Please download and install the Python 3.x version of
        Anaconda (Anaconda3). The FTP server is only accessible on campus.

        \noindent
        You can download LibreOffice from:
        \begin{itemize}
            \item Online: \url{http://www.libreoffice.org/download/libreoffice-fresh/}
            \item Campus FTP (Windows): \url{ftp://ftp.ee.up.ac.za/pub/windows/libreoffice}
            \item Campus FTP (Mac OSX): \url{ftp://ftp.ee.up.ac.za/pub/mac/libreoffice}
        \end{itemize}
        \textbf{Note:} See
        \url{http://www.libreoffice.org/get-help/install-howto/} for
        installation instructions. The FTP server is only accessible on campus.

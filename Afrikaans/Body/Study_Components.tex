\section{Studie Komponente}
        \begin{table}[!h]
             \begin{tabular}{|p{1.4cm}|l|p{2cm}|p{2cm}|}
                 \hline
                 {\bf Tema No.} & {\bf Onderwerp} & 
                    {\bf Leerure} & {\bf Kontak Sessies} \\
                 \hline
                 1  & Inleiding tot Rekenaars en Programmering &    &   \\
                 \hline
                 2  & Basiese Programmering                     & 12 & 3 \\
                    & \qquad Gebruik Python as 'n Sakrekenaar   &    &   \\
                    & \qquad Name en Voorwerpe                  &    &   \\
                 \hline
                 3  & Beheer Stellings                          & 24 & 6 \\
                    & \qquad Herhaling                          &    &   \\
                    & \qquad Vertakking                         &    &   \\                 
                 \hline
                 4  & Data Houer                             & 28 & 7 \\
                    & \qquad Lyste, ``tuples'', woordeboeke     &    &   \\
                    & \qquad Vektore en Matrikse                &    &   \\                 
                 \hline
                 5  & Gestruktureerde Programmering              & 24 & 6 \\
                    & \qquad Gebruiker-gedefinieerde Funksies   &    &   \\                 
                    & \qquad Kode Struktuur                     &    &   \\
                    & \qquad Lokale Veranderlikes               &    &   \\
                 \hline
                 6  & Sketse en Grafieke                        & 20 & 4 \\
                    & \qquad 2D Grafieke                        &    &   \\
                    & \qquad Grafiek Annotasie                  &    &   \\
                    & \qquad 3D Grafieke                        &    &   \\
                 \hline
                 7  & Dokument Hantering                        & 8  & 2 \\
                    & \qquad Lees en Skryf na Teks Dokument     &    &   \\                 
                    & \qquad Indeks Bestuur                        &    &   \\
                 \hline
                 8  & Ho\"{e} Vlak Programmering                & 12 & 3 \\
                    & \qquad Fout Hantering                     &    &   \\
                    & \qquad Addisionele Modules                &    &   \\    
                    & \qquad Gevorderde Programmering           &    &   \\
                 \hline
                 9  & Sigblaaie                                 & 28 & 7 \\
                    & \qquad Formules en Berekeninge            &    &   \\
                    & \qquad Sigblaai Inspekteur                &    &   \\
                    & \qquad Sketse en Grafieke                 &    &   \\
                    & \qquad Lini\^{e}re Programmering          &    &   \\
                    & \qquad Nie-lini\^{e}re Oplosser Probleme  &    &   \\
                    & \qquad Data Pilot                         &    &   \\
                    & \qquad Wat-as Situasie                    &    &   \\
                    & \qquad Visuele Formatering                &    &   \\
                 \hline
                 10 & HTML, Webtuistes  en die Internet         & 16 & 4 \\
                 \hline
                 11 & Databasisse                                & 8  & 2 \\
                 \hline
                    & {\bf TOTAAL}                              &180 &44 \\
                 \hline
             \end{tabular}
             \caption{Module Struktuur}
        \end{table}
        
    \subsection{Doel van die module}
        {\bf Gevorderde sigblaai toepassings:} \\
        Benoemde reekse, lini\^{e}re algebra,
        oplos van gelyktydige vergelykings, regressie, interpolasie,
        optimimeering en tabel manipulasie. \\ \\
        
        {\bf Basiese gestruktureerde programmering:} \\
        Herhaling, vertakking, sub-roetines, iterasies, Lees en skryf van data le\^{e}rs.
        Ontwikkeling, kodering en ontfouting van basiese programme in 'n ho\"{e} vlak
        progammeer taal. Programmeringsbeginsels word voorgestel deur wiskundige 
        konsepte soos limiete, differensiasie, integrasie in lini\^{e}re algebra. 
        Gestruktueerde programmering deur gebruik te maak van funksies en 
        beskikbare pakkette.  Basiese grafiese uitsette word ook gedek.
    
    \subsection{Module Struktuur}
	Die struktuur van die module word uiteengesit in die tabel op die vorige bladsy.
	Die voorgeskrewe ure vir die module sluit in: die kontaktyd (gedurende lesings and tutoriaale),
	selfstudie tyd sowel as voorbereiding van opdragte, toetse en 
	eksamens. Die wyse van onderrig is deur middel van lesings, tutoriale en opdragte.

    \subsection{Lesings Plan}
	Hierdie afdeling gee 'n uiteensetting van die lesing plan vir die semester.  Studente
	moet hierdie plan gebruik om seker te maak dat hulle nie agter raak met die
	lesings of tutoriale nie.
	
	Die tabel wat volg, lys in watter week die spesifieke studietemas gedek gaan word.
	Daar sal ongeveer elke week een totoriaal wees wat tydens daardie week voltooi moet word.
        
        \begin{table}[!h]
            \begin{center}
             \begin{tabular}{|l|l|l|l|l|}
                 \hline
                 {\bf Week} & {\bf Lesings} & {\bf Datums} & {\bf Studie Tema} & {\bf Tutoriaal} \\
                 \hline
                 1  & 4     & 11 –- 15 Feb     & 1, 2  & Spesiaal \\
                 2  & 4     & 18 –- 22 Feb     & 3     & T1 \\
                 3  & 4     & 25 Feb –- 1 Mar  & 3, 4  & T2 \\
                 4  & 4     & 4 –- 8 Mar       & 4     & T3 \\
                    &       & == TOETSWEEK 1 == &       & \\
                 5  & 2     & 18 -- 19 Mar     & 4     & T4 \\
                    &       & ==== VAKANSIE ==== &       & \\
                 6  & 3 / 4 & 2 -- 5 Apr       & 5     & T4 \\
                 7  & 4     & 8 -- 12 Apr      & 5, 6  & T5 \\
                 8  & 4     & 15 -- 19 Apr     & 6     & T6 \\
                 9  & 4     & 22 -- 26 Apr     & 9     & T7 \\
                 10 & 3 / 2 & 29 Mar -- 3 Mei  & 9     & T8 \\
                    &       & == TOETSWEEK 2 == &       & \\
                 11 & 4     & 13 -- 17 Mei     & 7, 11 & T9 \\
                 12 & 4     & 20 -- 24 Mei     & 10    & T10 \\
                 13 & 4     & 27 -- 30 Mei     & 8     & T11 \\
                    &       & ===== Eksamen ===== &       & \\
                 \hline
             \end{tabular}
             \caption{Lesings Plan}
            \end{center}
        \end{table}
        
	Geselekteerde oefenprobleme uit die studienotas, sal gedek word tydens
	die tutoriaal sessies.  Geselekteerde hoofstukke en probleme (uit die 
	studienotas) wat tydens die tutoriaal sessie gedek word sal beskikbaar
	gestel word op ClickUP.
        
	NOTA:  Die ``spesiale'' tutoriaal sessie in die eerste week is net 
	vir studente wat die kursus herhaal. Dit is noodsaaklik vir studente
	wat die kursus herhaal om die eerste lesings van die semester en die
	``spesiale'' tutoriaal sessie by te woon.  Die programmeertaal wat 
	laasjaar gebruik was (Octave) word vervang die jaar met Python en die
	``spesiale'' tutoriaal sessie sal gebruik word om die verskille tussen 
	die twee te bespreek asook die fundamentele verskille in die gedrag van Python 
	te beklemtoon.
        
    \subsection{Notas}
	Voorgeskrewe ure sluit kontaktyd in, asook die beraamde tyd nodig vir die
	voorbereiding van toetse en eksamens.  Kontaksessies dui die 
	gere\'{e}lde lesings aan.  Die aantal lesings per hoofstuk is tentatief.  Dit
	kan verander afhangende van die vordering tydens die semester.
        
	Let wel dat van die onderafdelings in die handboek-hoofstukke nie
	gedek gaan word in die lesings nie. Tog mag die onderafdelings gegee
	word vir selfstudie.  Die dosent sal informasie verskaf oor die 
	onderafdelings wat nie gedek gaan word vir eksamen en toets doele\"{i}ndes nie.
    
    \subsection{Studie Tema Beskrywings}
        \subsubsection{Tema 1: Inleiding tot Rekenaars en Programmering}
	    Die inleiding stel die student in staat om 'n algemene oorsig te kry.
	    Hierdie afdeling sal nie uitdruklik getoets word nie.  Die begrip
	    van rekenaarargitektuur en vloeidiagramme sal die volgende afdelings
	    meer toeganlik maak.
            
        \subsubsection{Tema 2: Basiese Programmering}
            \paragraph{Gebruik van Python as 'n Sakrekenaar:}
		 Dit student moet verseker dat hy/ sy gemaklik is met die 
		 \emph{IPython Console} en \emph{Spyder} omgewings.  Dit is die 
		 student se verantwoordelikheid om genoeg tyd in hierdie omgewings
		 te spandeer gedurende die semester.
            \paragraph{Name en Voorwerpe:}
		 Die student moet vetroud wees met die afdelings in die klasnotas wat handel oor
		 name en voorwerpe b.v. voorwerpe word geskep in geheue en word
		 gebonde aan 'n naam.  Die student moet ook vertroud wees met die 
		 geheue-model in Python.  Die student moet ook die veskil weet
		 tussen verskillende voorwerp tipies b.v. int, float, str, bool.

                
        \subsubsection{Tema 3: Beheer Stellings}
            \paragraph{Herhaling:}
                Die student moet in staat wees om te onderskei tussen onvoorwaardelike
                ('for') en voorwaardelike ('while') lusse vir die doel van probleemoplossing. 
                Die student moet bekend wees met Python se sintaks en indentasie-re\"{e}ls en moet in staat wees om 
                voorwaardelike en onvoorwaardelike herhalings te kan implimenteer.
                
            \paragraph{Vertakking:}
		Die student moet in staat wees om vertakking te identifiseer vanaf
		'n probleem stelling. Die student moet bekend wees met Python se sintaks
		en indentasie om vertakking effektief te implimenteer om 'n gegewe
		probleem op te los.
            
        \subsubsection{Tema 4: Data Hanteerders}
            \paragraph{Lyste, ``tuples'' and woordeboeke:}
		 Die student moet die beginsels van lyste, ``tuples'' en woordeboeke
		 verstaan sowel as die verwante Python sintaks.  Die student
		 moet in staat wees om lyste, ``tuples'' en woordeboeke te identifiseer
		 en te implimenteer.  Die student moet in staat wees om 'n 
		 bestaande program wat veranderlikes gebruik te herskryf met
		 behulp van of lyste, ``tuples'' of woordeboeke.
                
            \paragraph{Vektore en Matrikse:}
		 Die student moet die beginsels van vektore en matrikse verstaan 
		 sowel as die verwante Python sintaks.  Die student moet in staat wees
		 om vektore en matrikse te implimenteer.  Die student moet die 
		 basiese matriks- en vektor-bewerkings verstaan en kan implimenteer
		 in 'n program.  Die student moet in staat wees om 'n bestaande
		 program wat veranderlikes gebruik te herskryf na 'n program wat of vektore of 
		 matrikse gebruik.
		 
        \subsubsection{Tema 5: Gestruktueerde Programmering}
	     Die studen moet vertroud wees met funksies, sub-funksies en kode 
	     struktuur.  Die student moet in staat wees om funksies te skryf
	     met verskeie insette en uitsette sowel as opsionele (kernwoord) insette.
	     Die student moet in staat wees om 'n bestaande program te herskryf
	     sodat dit gebruik maak van funksies en sub-funksies.  Die student
	     moet vertroud wees met lokale veranderlikes en moet hulle omvang 
	     binne funksies, sub funksies of modules verstaan.
            
        \subsubsection{Tema 6: Sketse en Grafieke}
	     Die student moet in staat wees om 2D grafieke te skep d.w.s. die student
	     moet vertroud wees met verskeie voorstellings van data, die aanbieding
	     van menigte grafieke op een figuur, die voorstelling van roosters, 
	     benoeming van asse, benoeming van figure, die gebruik van legendes
	     en die skaal van asse.  Die student moet in staat wees om 3D figure
	     te skep.  Die student moet vertroud wees met die skep van 
	     oppervlak en ``mesh'' grafieke.

            
        \subsubsection{Tema 7: Data Hantering}
	     Die student moet vertroud wees met die lees en skryf na en vanaf
	     le\^{e}rs, sowel as die verskillende le\^{e}r tipes in Python.  Die 
	     student moet in staat wees om data te stoor in 'n gegewe formaat.
	     Die student moet ook verstaan hoe om toegang te kry tot verskillende 
	     indekse in Python.
	     

        
        \subsubsection{Tema 8: Ho\"{e} Vlak Programmering}
            Die student moet in staat wees om oorbepaalde en onderbepaalde 
            lini\^{e}re, nie-lini\^{e}re, en lini\^{e}re differensiaal stelsels van vergelykings op te los
            deur gebruik te maak van addisionele modules in Python. Die student
            moet in staat wees om polinome te manipuleer, optimeering uit te voer, 
            numeries te integreer, beskrywende statistiese maatstawwe te verkry
            en numeriese data te interpoleer deur gebruik te maak 
            van addisionele modules in Python.
                
            Die student moet in staat wees om in Python se dokumentasie
            rond te soek (snuffel), sowel as die aanlyn dokumentasie. 
            Verder moet die student in staat wees om dokumentasie, van verskeie funksies
            te vind, lees en verstaan om ten einde
            'n spesifieke probleem op te los.  Die student moet dus in staat wees 
            om onafhanklik hul kennis in Python te verhoog.

        \subsubsection{Tema 9: Sigblaaie}
            \paragraph{Formules en Berekeninge:}
                Die student moet in staat wees om probleme op te los deur gebruik 
                te maak van formules. Verder moet die student in staat wees om probleme
                op te los deur gebruik te maak van bekende funksies sowel as die
                funksie ``wizard'' vir onbekende funksies.  Laastens moet die student in staat
                wees om die reekse van selle te benoem en te gebruik in berekeninge.
                
		Die student moet in staat wees om filters te gebruik om data te filter
		en berekeninge te verrig op di\'{e} datastel.  Die student
		moet in staat wees om berekeninge te verrig met data wat verdeel is
		tussen verskillende sigblaaie.
		
            \paragraph{Sigblaai Inspekteur:}
		Studente moet in staat wees om die sigbladinspekteur te gebruik
		om ten einde hulself vertroud te maak met die afhanklikhede in 'n 
		onbekende sigblad.  Die student moet in staat wees om te kan omskakel 
		tussen formule- en waardesig.
  
            \paragraph{Sketse en Grafieke:}
                Die student moet in staat wees om grafieke te skep en te verander.
                Die student moet ook tendense in die data kan vind.          

            \paragraph{Lini\^{e}re Programmering:}
                Die student moet in staat wees om lini\^{e}re programmeringsprobleme
                op te los met behulp van LibreOffice Calc. Die student
                moet ook in staat wees om probleme op te los met ``integer'', ``real''
                en ``boolean'' veranderlikes.  Die student moet ook in staat wees om 
                verskeie beperkinge te akkommodeer.

            \paragraph{Nie-Lini\^{e}re Oplosser:}
                Die student moet in staat wees om 'n een dimensionele nie-lini\^{e}re 
                probleem op te los deur gebruik te maak van LibreOffice Calc se 
                ``goal seek'' funksie.

            \paragraph{Data ``Pilot'':}
		Die student moet in staat wees om verwante data uit te trek 
		deur gebruik te maak van die Data ``Pilot'' in LibreOffice Calc.

            \paragraph{Wat-as Situasie:}
                Die student moet in staat wees om 'n `wat-as' situasies op te los 
                deur gebruik te maak van die 'scenario'-funksie  in LibreOffice Calc.

            \paragraph{Visuele Formatering:}
                Die student moet in staat wees om die uitleg van 'n sigblad te verbeter
                deur gebruik te maak van kleure, beklemtooning (verhelder), grense en lettertipe
                eienskappe.  Die student moet in staat wees om 'conditional formatting' 
                van selle uit te voer.
        
            
        \subsubsection{Tema 10: HTML, Webtuistes en die Internet}
		 Die student moet besef dat 'n webtuiste basies net 'n teks le\^{e}r
		 is met spesiale HTML ``tags''.  Die student moet in staat wees om 
		 bestaande HTML webtuistes te verander.  Ten slotte, moet die student
		 weet hoe om gevorderde soektogte op die internet te doen vir
		 informasie.
        

        \subsubsection{Tema 11: Databasisse}
		 Die student behoort te weet en te verstaan wat 'n databasis is en 
		 wanneer 'n databasis die regte datastruktuur is om data te stoor.
		 Die student moet in staat wees om 'n databasis te skep van informasie
		 deur gebruik te maak van sagteware. Die student moet ook in staat 
		 wees om navrae en verslae te skep wat gebasseer is op die data in die 
		 databasis.
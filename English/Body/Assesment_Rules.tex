\section{Rules of Assessment}
    Refer to the exam regulations in the Yearbooks of the Faculty of Engineering,
    Built Environment and Information Technology.

    To pass the subject a student must:
    \begin{itemize}
        \item Obtain a final mark of at least 50\% \\ {\bf and}
        \item Obtain a subminimum of 40\% for the semester mark \\ {\bf and}
        \item Obtain a subminimum of 40\% for the final examination
%         \item Obtain a subminimum of 50\% for the ECSA exit outcome 5 
%               assessment matrix for semester test 1, semester test 2, 
%               sick test (if applicable), exam and re-exam (if applicable).
    \end{itemize}
    
    \subsection{Determination of Final Mark}
        The final mark is compiled as follows:
        \begin{itemize}
            \item Semester mark: 50\%
            \item Final Exam mark: 50\% (3-hour exam), closed-book
        \end{itemize}

    \subsection{Determination of Semester Mark}
        The semester mark will be determined as shown in the table below:
        \begin{table}[!h]
            \begin{center}
             \begin{tabular}{|p{5cm}|c|l|l|}
                 \hline
                 {\bf Evaluation Method} & {\bf No. of} & 
                 {\bf Contribution of ea.} & {\bf Total} \\
                 \hline
                 Semester tests, written, closed book 
                    & 2 & 40\% & {\bf 80\%} \\ \hline
                 Semester Project 
                    & 1 & 20\% & {\bf 20\%} \\
                 \hline
                 \multicolumn{3}{|l|}{{\bf Total}} & {\bf 100\%} \\
                 \hline
             \end{tabular}
             \caption{Determination of Semester Mark}
            \end{center}
        \end{table}
    
    \subsection{Semester Tests}
        Two tests will be written during the semester, in the weeks 9 to 16 March
        2013 and 4 to 11 May 2013. The duration of each test will be 90 minutes.
        Syllabi of the tests will be announced during the lecture week preceding 
        the test week. Both tests will be closed-book.
        
        Memoranda on the scheduled tests will be made available in electronic 
        format on Click-UP and will not be discussed during lectures. You may
        make use of the discussion board on Click-UP should you have a specific
        question regarding a problem given in the semester tests.

    \subsection{Appeals and queries on marks}
        The marks awarded for assignments and semester tests will be
        posted on Click-UP. If you have appeals or queries regarding marks
        you must follow the procedure outlined below within 14 days from
        receiving the marks. After the 14 days no marks will be altered.

        I will wait until I have received all the question papers from
        students and until the 14 days have passed before looking at the
        papers again.

        \subsubsection{Appeal Process}
            \begin{enumerate}
                \item Download the memorandum from Click-UP and mark the relevant
                      question in pencil.
                \item Add up the marks for the question.
                \item Write down the question number on the top of the cover of 
                      your question paper.
                \item Hand your question paper in to the lecturer at the end of
                      the lecture.
            \end{enumerate}

            I will then just check whether or not you marked your paper
            correctly and sign next to it if it is indeed correct.
        
        \subsubsection{Absence from Test(s) and/or Exams}
            Refer to the Study guide of the Department.
        
        \subsubsection{Plagiarism}
            Refer to the Study guide of the Department.




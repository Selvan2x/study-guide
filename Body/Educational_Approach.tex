\section{Opvoedkundige Benadering}
    %Die belangrikheid van programmeringsvaardighede is vinnig besig om net 
    %so belangrik te word vir ingenieurs as hulle vaardigheid in wiskunde.  Die 
    %rede vir dit is die feit dat rekenaars die sakrekenaars van die toekoms is.
    %Vir 'n ingenieur om effektief 'n rekenaar te gebruik, moet hy ten minste 
    %bietjie vaardigheid opbou in programmering.
    Programmeringskennis en -vaardighede is belangrik vir ingenieurs. Komplekse 
    berekeninge en analises, wat vermoeiend en tydsaam sal wees indien dit met die 
    hand gedoen word, kan binne sekondes opgelos word deur gebruik te maak van 'n 
    goed gestruktureerde program.

    Die algemene doelwitte vir hierdie module is om die student in staat te stel om:
    \begin{itemize}
        \item ingenieursprobleme op te los
	      deur ontwikkeling en ontfouting ('debugging') van basiese 
	      rekenaarprogramme.
        \item rekenaar data te verwerk en uit 
	      te voer as informasie of kennis (visueel of teks).
    \end{itemize}

    Element\^{e}re wiskundige konsepte, waarmee die studente reeds vertroud 
    moet wees, sal gebruik word om die basiese rekenaarlogika en 
    programmeringsbeginsels 
    te illustreer. Daar sal van die student verwag word om 
    wiskundige formules en/of beginsels te vertaal na 'n werkende 
    rekenaarprogram
    om ander probleme mee op te los.

    Die opvoedkundige benadering is dus probleem gedrewe, m.a.w 
    programmeringsvaar-dighede 
    word vekry deur die oplos van probleme. Die bemeestering van die
    tutoriale en huiswerk probleme is noodsaaklik om sukselvol te wees in hierdie
    module.
    
    Student geori\"{e}nteerde en ko\"{o}peratiewe studiemetodes sal toegepas word 
    gedurende die lesings en tutoriaalsessies om ten einde die kernkonsepte
    van die kursus oor te dra.  Daar word van studente verwag om deel te neem 
    aan klasbesprekings. Hierdie klasbesprekings skep 'n geleentheid om ervarings te deel
    en probleme op te los as deel van 'n span.  Dit sal 'n idee gee van wat 
    algemeen voorkom in die industrie. Probleem geori\"{e}nteerde sessies tydens die lesings
    sal die studente die geleentheid gee om moeilike konsepte te 
    assimileer en verstaan.

    \subsection{Departementele Studiegids}
        Die studiegids is 'n belangrike deel van die algemene studiegids van die 
        Departement. In die Departementele studiegids word informasie gegee
        oor die visie en missie van die departement, algemene administrasie en 
        regulasies (professionaliteit en integriteit, kursus verwante inligting
        en formele kommunikasie, werkswinkel gebruik en veiligheid, plagiaat,
        klasverteenwoordiger pligte, sieketoets en siek eksamenriglyne, vakansiewerk,
        universiteitsregulasies, vrae wat dikwels gevra word), ECSA uitkomste
        en ECSA uittreevlak uitkomste, ECSA kennisarea, CDIO, nuwe kurrikulum en 
        assessering van kognitiewe vlakke.  Daar word verwag dat die student baie vetroud word
        met die inhoud van die Departementele Studiegids.  Dit is beskikbaar in Engels
        en Afrikaans in die
        \href{http://web.up.ac.za/default.asp?ipkCategoryID=11426&subid=11426&%
              ipklookid=7}{Departement se webtuiste.\footnote{%
              \url{http://web.up.ac.za/default.asp?ipkCategoryID=11426&%
                   subid=11426&ipklookid=7}}}
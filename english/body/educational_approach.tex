\section{Educational Approach}
Programming knowledge and skills are important for engineers. Complex
and repetitive calculations and analyses that would take days if done
by hand can take only seconds with a well structured program. Although
the program itself takes time to code, but once coded allows the
engineer to ask different questions and conduct various investigations
that would otherwise not be feasible or possible. Programming combined
with mathematics has become a prerequisite to basic engineering
literacy and engineering problem solving. In addition the life cycle
of a computer program is similar to the design of a product or the
design of a process.

It is for these reasons that engineers obtain at least some
proficiency in programming. The general objectives of this module are:
    \begin{itemize}
        \item To enable the students to solve engineering problems by
            developing, debugging, and running basic computer programs.
        \item To enable the students to process computer data and output
            (text and visual) information or knowledge.
    \end{itemize}

    Elementary mathematical and physics concepts, which the students
    should already be familiar with, will be used to illustrate basic
    computer logic and programming principles. The student will be
    required to 'translate' mathematical formulas and/or principles
    into working computer programs to solve some problems.

    The \textbf{educational approach} is \textbf{problem driven}, in
    other words programming skills are acquired by solving
    problems. The mastering of tutorial and homework problems is
    therefore essential to be successful in this module. Similar to
    learning a natural language that requires dedicated time
    conversing, programming requires dedicated clock time in front of
    a computer solving engineering problems. \textit{Ensure that you sit in
    front of a computer solving engineering problems using Python on a
    daily basis.}

  Student-orientated and cooperative study methods will be applied
  during the lectures and tutorial sessions in order to establish the
  core concepts of the course. Lectures will strongly focus on solving
  problems in class and developing the programming concepts as
  required to solve mathematical and physics problems.

  Students are expected to participate in discussions in class since
  this creates an opportunity to share experiences and solve problems
  in a team orientated environment. This will mimic what generally
  occurs in industry. Problem solving sessions during lectures will
  provide students with opportunities to assimilate and understand
  difficult concepts. Students are encouraged to bring their laptops
  to the lectures to code the covered material in class.

\section{Studie Komponente}
    \subsection{Doel van die module}
        {\bf Gevorderde sigblaai toepassings:} \\ Benoemde reekse, line\^{e}re
        algebra, oplos van gelyktydige vergelykings, regressie, interpolasie,
        optimimeering en tabel manipulasie.

        {\bf Basiese gestruktureerde programmering:} \\ Herhaling, vertakking,
        sub-roetines, iterasies, Lees en skryf van data le\^{e}rs.
        Ontwikkeling, kodering en ontfouting van basiese programme in 'n
        ho\"{e} vlak progammeer taal. Programmeringsbeginsels word voorgestel
        deur wiskundige konsepte soos limiete, differensiasie, integrasie in
        line\^{e}re algebra.  Gestruktueerde programmering deur gebruik te maak
        van funksies en beskikbare pakkette.  Basiese grafiese uitsette word
        ook gedek.

    \subsection{Module Struktuur}
        Die struktuur van die module is uiteengesit in
        Tabel~\ref{tab:study_comp}.  Die voorgeskrewe ure vir die module sluit
        in: die kontaktyd (gedurende lesings and tutoriaale), selfstudie tyd
        sowel as voorbereiding van opdragte, toetse en eksamens. Die wyse van
        onderrig is deur middel van lesings, tutoriale en opdragte.

        \begin{longtable}{|p{1.4cm}|p{8.4cm}|p{2.2cm}|p{1.8cm}|}
             \hline
             \multicolumn{1}{|p{1.4cm}|}{\textbf{Tema No.}} & %
             \multicolumn{1}{p{8.4cm}|}{\textbf{Onderwerp}} &  %
             \multicolumn{1}{p{2.2cm}|}{\textbf{Leerure}} & %
             \multicolumn{1}{p{1.8cm}|}{\textbf{Kontak Sessies}} \\
             \hline \endfirsthead

             \multicolumn{4}{l}%
             {\textbf{...vervolg van vorige bladsy}} \\
             \hline
             \multicolumn{1}{|p{1.4cm}|}{\textbf{Tema No.}} & %
             \multicolumn{1}{p{8.4cm}|}{\textbf{Onderwerp}} &  %
             \multicolumn{1}{p{2.2cm}|}{\textbf{Leerure}} & %
             \multicolumn{1}{p{1.8cm}|}{\textbf{Kontak Sessies}} \\
             \hline

             \hline \endhead
             \hline \endfoot
             \endlastfoot

             1  & Inleiding tot Rekenaars en Programmering     &    &    \\
             \hline
             2  & Basiese Programmering                        & 12 & 4  \\
                & \qquad Gebruik Python as 'n Sakrekenaar      &    &    \\
                & \qquad Name, Voorwerpe en Toewysing          &    &    \\
                & \qquad Ruil van Toegewysde Voorwerpe         &    &    \\
                & \qquad Basiese Lyste                         &    &    \\
             \hline
             3  & Beheerstellings                              & 56 & 12 \\
                & \qquad Iterator Gebaseerde Lus  (For Lus)    &    &    \\
                & \qquad Voorwaardelike Lus  (While Lus)       &    &    \\
                & \qquad Vertakking (If, Elif, Else)           &    &    \\
             \hline
             4  & Lyste Voortgeset                             & 8  & 2  \\
                & \qquad Itereer deur Lyste                    &    &    \\
                & \qquad Dinamiese Lys Groei + Lusse           &    &    \\
             \hline
             5  & Gestruktureerde Programmering (Funksies)     & 8  & 2  \\
                & \qquad Gebruiker-gedefinieerde Funksies      &    &    \\
                & \qquad Kode Struktuur  (modules vs scripts)  &    &    \\
                & \qquad Lokale Naamruimte                     &    &    \\
             \hline
             6  & Geneste Stelling                             & 20 & 4  \\
                & \qquad Geneste Beheerstellings               &    &    \\
                & \qquad Geneste Iterators                     &    &    \\
                & \qquad Vereenvoudigings met Funksies         &    &    \\
             \hline
             7  & Sketse en Grafieke                           & 16 & 4  \\
                & \qquad Numpy Funksies en Voorwerpe           &    &    \\
                & \qquad Grafiese Annotasie                    &    &    \\
                & \qquad 3D Grafieke                           &    &    \\
             \hline
             8  & Dokument Hantering                           & 12 & 3  \\
                & \qquad Lees en Skryf van Data na 'n L\^eer   &    &    \\
                & \qquad Data beskikbaarheid tussen Programme  &    &    \\
             \hline
             9  & Ho\"{e} Vlak Programmering                   & 10 & 3  \\
                & \qquad Addisionele Modules                   &    &    \\
                & \qquad Gevorderde Ingeboude Funksies         &    &    \\
             \hline
             10 & Grafiese Gebruikerskoppelvlakke              & 6  & 2  \\
                & \qquad TKinter Module                        &    &    \\
                & \qquad Vorms en Vormvoorwerpe                &    &    \\
                & \qquad Muis en Sleutelbord Gebeure           &    &    \\
             \hline
             11 & Sigblaaie                                    & 32 & 8  \\
                & \qquad Formules en Berekeninge               &    &    \\
                & \qquad Sigblaai Inspekteur                   &    &    \\
                & \qquad Sketse en Grafieke                    &    &    \\
                & \qquad Line\^{e}re Programmering             &    &    \\
                & \qquad Nie-line\^{e}re Probleme              &    &    \\
                & \qquad Data Pilot                            &    &    \\
                & \qquad Wat-as Situasie                       &    &    \\
                & \qquad Visuele Formatering                   &    &    \\
             \hline
                & {\bf TOTAAL}                                 &180 & 44 \\
             \hline
         \caption[Module Struktuur]{Module Struktuur} \label{tab:study_comp}
        \end{longtable}

    \subsection{Lesings Plan}
        Die lesingsplan vir die semester volg.  Daar is 'n totaal van 48
        lesings en 12 tutoriaalsessies oor die 13 weke van die semester. Maak
        gebruik van hierdie plan om seker te maak dat jy nie agter raak nie in
        terme van die lesings en tutoriale nie.

        Tabel~\ref{tab:lec_plan} sit uiteen hoe die studietemas deur die loop
        van die semester sal volg.  Elke week het 'n ooreenstemmende tutoriaal
        wat deur die loop van die week voltooi en teen die	einde van die week
        ingehanding moet word. Die tutoriaalsessies is om die studente te help
        om die materiaal te bemeester om sodoende die tutoriale te kan voltooi.

        \begin{table}[!h]
            \begin{center}
             \begin{tabular}{|l|l|l|l|l|}
                 \hline
                 {\bf Week} & {\bf Lesings} & {\bf Datums} & {\bf Studie Tema}
                 & {\bf Tutoriaal} \\
                 \hline
                 1  & 4     &  1 –-  5 Feb      & 1, 2  & Geen \\
                 2  & 4     &  8 –- 12 Feb      & 3     & T1 \\
                 3  & 4     & 15 –- 19 Feb      & 3     & T2 \\
                 4  & 4     & 22 –- 26 Feb      & 3     & T3 \\
                    &       & == TOETSWEEK 1 == &       & \\
                 5  & 4     &  7 –- 11 Mar      & 4, 5  & T4 \\
                 6  & 4     & 14 -- 18 Mar      & 6     & T5 \\
                    &       & ==== RECESS ====  &       & \\
                 7  & 4     &  4 --  8 Apr      & 11    & T6 \\
                 8  & 4     & 11 -- 15 Apr      & 11    & T7 \\
                    &       & == TOETSWEEK 2 == &       & \\
                 9  & 2     & 25 -- 29 Apr      & 7     & T8 \\
                 10 & 4     &  3 --  6 Mei      & 7, 8  & T9 \\
                 11 & 4     &  9 -- 13 Mei      & 8, 9  & T10 \\
                 12 & 4     & 16 -- 20 Mei      & 10, Herhaling & T11 \\
                 13 & 2 / 4 & 23 -- 25 Mei      & Herhaling & T12 \\
                    &       & ===== EKSAMEN ===== &       & \\
                 \hline
             \end{tabular}
             \caption{Lesings Plan} \label{tab:lec_plan}
            \end{center}
        \end{table}


	Veronderstelde ure kontaktyd, asook die beraamde tyd 
         nodig is vir voorbereiding vir toetse en eksamens. Kontak sessies dui 
         op die gereelde lesings. Die aantal kontaksessies per hoofstuk is 
         tentatief, en mag verander afhangende van die vordering tydens die lesings.

	Let asseblief daarop dat sommige onderafdelings in die notas nie tydens die lesings
	gedek word nie. Sommige onderafdelings sal ook as selfstudie gegee word.
 	Onderafedlings wat nie vir toets of eksamen doeleindes is nie sal op \textit{ClickUP} beskikbaar 
	gemaak word.

	Die tutoriaalsessies is beskikbaar om studente te help om die materiaal te bemeester sodat 
	gekose oefenprobleme uit die studienotas opgelos kan word. Die oefenprobleme sal op 	   
     	\textit{ClickUP} aangekondig word.	
        
	\subsection{Fundamentele Konsepte}
	Bemeestering van die volgende konsepte in praktiese probleemoplossing is 'n vereiste om die
  	module te slaag. Indien enige
	van die volgende konsepte ontbreek sal die student nie die module slaag nie:

        \begin{itemize}
	\item Basiese voorwerpe (bv. int teenoor float)
            \item Die invoer en korrekte gebruik van modules
            \item Die Iterator lus (for  lus) (behoorlike gebruik en begrip wanneer en hoekom te gebruik
                'n lus vir)
            \item Iterasie deur lyste en die groei van lyste (enkele lyste , nie
                geneste lyste)
            \item Voorwaardelike lus (while lus) (korrekte gebruik; begrip be\"eindiging
                voorwaardes; begrip vir die gebruik en toepassing van 'n voordelike lus)
            \item Vertakkings (if-elif-else) (korrekte gebruik; begrip oor die program vloei en
                beheer as ook wanneer en hoekom 'n if-elif-else gebruik word)
            \item funksies (in staat om 'n nuwe funksie behoorlik te skep; verstaan oor
                hoe voorwerpe na en van funksies gestuur en ontvang word; in staat wees om enige
                geskepte funksie behoorlik te gebruik)
            \item File lees en skryf van \emph{CSV} lêers.
            \item Plot (in staat wees om eenvoudige 2D grafieke te skep met behoorlike
                byskrifte)
            \item Die behoortlike gebruik van sigblaaie vir data verwerking van groot data stelle.
        \end{itemize}
    
    \subsection{Studie Tema Beskrywings}
        \subsubsection{Tema 1: Inleiding tot Rekenaars en Rekenaarprogrammering}

	Die student moet in staat wees om Python te gebruik eenvoudige wiskundige te doen 
             berekeninge en verseker dat hy / sy gemaklik is met 
             die \emph{IPython Console} en \emph {Spyder} omgewings. Dit is die student se verantwoordelikheid om genoegsame tyd gedurende die semester agter die rekenaar te bestee om te verseker dat hy / sy gemaklik is
	met die omgewings. Die student moet Python se ingebou \texttt{math} module kan laai
            en die funksies kan gebruik. Die student moet die ingeboude hulp kan gebruik om meer inligting oor 
	'n module of die funksies binne 'n module te kan kry.
            
        \subsubsection{Tema 2: Basiese Programmering}
            \paragraph{Gebruik van Python as 'n Sakrekenaar:}
		 Dit student moet verseker dat hy/ sy gemaklik is met die 
		 \emph{IPython Console} en \emph{Spyder} omgewings.  Dit is die 
		 student se verantwoordelikheid om genoeg tyd in hierdie omgewings
		 te spandeer gedurende die semester.
       
	     \paragraph{Name, Voorwerpe en Toewysing:}
		 Die student moet vetroud wees met die afdelings in die klasnotas wat handel oor
		 name en voorwerpe b.v. voorwerpe word geskep in die rekenaar se geheue en 
		 gebind aan 'n naam.  Die student moet ook vertroud wees met die 
		 geheue-model in Python dat 
		 Die student moet ook die veskil weet
		 tussen verskillende voorwerp tipies bv. int, float, str, bool asook vertroud wees met 
		die gebruik van \texttt{type()} om die tipe van 'n voorwerp te bepaal.

	\paragraph{Ruil van voorwerpe:}
		Die student moet in staat wees om die name gebind aan twee voorwerpe te ruil sonder 
             	om enig van die voorwerpe te verloor. Die student het ook te wees 
            	 vertroud is met die geheue model oor hoe die name gebind
	          aan die voorwerpe ruil.
	
	  \paragraph{Basiese Lys Generasie:}
                  Die student moet in staat wees om eenvoudige lys voorwerpe te genereer, 
             beide deur die eksplisiete definisie van elke voorwerp in die lys of deur middel van 
	   die \texttt{range} funksie gebruik te maak. Die student 
             moet ook vertroud met die verskillende lys operateurs wees 
             \texttt {(+, *)}, funksies (\texttt{sorted}) en \textit{inplek}
             funksies (metodes) 
             \texttt {append,insert,index,sort,reverse}). Die student 
             moet Python se ingeboude hulp verstaan en kan gebruik
		 om inligting oor verskillende funksies en 
             \textit{inplek} funksies te bekom.

                
        \subsubsection{Tema 3: Beheer Stellings}
            \paragraph{Iterator Gebaseerde Lus (For lus):}
		Die student moet verstaan ​​dat die \textit{for lus} 
             itereer deur iterators (bv. die elemente van 'n 
             lys). Die student moet in staat wees om te identifiseer wanneer en waarom
             \textit{for lusse} gebruik  word, van 'n gegewe probleem 
             stelling. Die student moet vertroud wees met Python se  
             sintaksis en inkeping. Die student moet gemaklik wees 
             \textit {for lusse} in 'n program te implementeer om gepaste probleme mee op te los.

	     \paragraph{Voorwaardelike Lus (While lus):}
		Die student moet voorwaardelike stellings verstaan asook hoe
             om hulle te kombineer met \texttt {and} en \texttt{or} 
             operateurs. Die \texttt{while} lus itereer so lank as wat die voorwaardelike stelling
    	    wat ge\"evalueer word \texttt{waar} of \texttt{1} is. Die 
             student moet verstaan ​​dat indien die stellings nie 
             behoorlik opgedateer word tydens elke iterasie nie, 'n \textit {while lus} in 'n oneindige lus
             vasgevang kan raak, waar (\texttt {Ctrl-C} uit die lus breek en die program evaluasie stop. Die 	  	
	    student moet in staat wees om te 
             identifiseer wanneer en waarom om 'n terwyl lus benodig word vanuit 'n gegewe 
             probleemstelling. Die student moet vertroud wees met Python se  
             sintaksis en inkeping. Die student moet gemaklik wees 
             \textit {while lusse} in program te implementeer om gepaste probleme mee op te los.
 	    Die student moet gemaklik wees om geskikte voorwaardelike stellings op te stel vir gebruik met
	    \textit{while lusse} om gepaste probleme mee op te los.
		
            \paragraph{Vertakking:}
		Die student moet voorwaardelike stellings verstaan asook hoe
             om hulle te kombineer met \texttt {and} en \texttt{or} 
             operateurs. Die student moet in staat wees die 
             gebruik van takke (if-elif-else)
             verklaring) van 'n probleemstelling te identifiseer. Die student moet vertroud wees met Python se  
             sintaksis en inkeping. Die student moet gemaklik wees 
             \textit {if-elif-else vertakkings} in program te implementeer om gepaste probleme mee op te los. 
	   Die student moet verstaan ​​hoe elke \textit{if-elif-else} die vloei en beheer van 'n program affekteer.	
	  Die student moet gemaklik wees om geskikte voorwaardelike stellings op te stel vir gebruik met
	    \textit{if-elif-else vertakkings} om gepaste probleme mee op te los.
            
	\subsubsection {Tema 4: Iterators (Lyste)} 
         Die student moet die beginsels van die lyste verstaan asook die Python sintaksis daarvan. 
	Die student moet verstaan dat 'n lys 'n meervuldige voorwerp houer is. Die student moet instaat
	wees om lyste te skep en groei, asook om toegang of veranderinge aan elemente in 'n lys aan te
	bring. Die student moet gemaklik wees om \textit{lyste} in 'n program te gebruik om gepaste probleme 	mee op te los. Die student moet instaat wees om programme wat meervoudige enkel element 	 
	voorwerpe bevat herskryf om met lyste te werk. Die student moet kan bepaal wanneer dinamiese
	groei van lyste nodig is en dit kan implementeer om gepaste probleme mee op te los.

	\subsubsection {Tema 5: Gestruktureerde Programmering (Funksies)} 
        Die student moet vertroud met funksies en die kode struktuur van funksies. 
	Die student moet behoorlike funksies kan skep en 
         gebruik wat verskeie insette en uitsette neem. Die student 
        moet  'n bestaande program kan herskryf sodat dit gebruik maak van van self geksryfde funksies. 
	Die student moet vertroud wees 
        met plaaslike naamruimtes en die verstaan ​​"naam omvang" binne 'n 
        funksie, module en script verstaan. Die student moet verstaan dat enige
        voorwerp 'n inset of uitset van 'n funksie kan wees. Dit sluit ander funksies in wat insette of uitsette 
	van 'n funksie kan wees. Bemeestering van hierdie konsep is veral belangrik vir die gebruik van funksies in 
	numeriese modules. 

\subsubsection {Tema 6: Geneste Strukture} 
	             Die student moet in staat wees om 'n komplekse probleem af te breek in logiese beheer strukture (for lus, while lus, if-elif-else vertakking). Die 
             student moet in staat wees om te bepaal watter geneste beheerstrukture is nodig 
             van die onteleding van die probleem stelling. Die student moet dit dan kan implementeer
	   in 'n program om die gegewe 
             probleem op te los. Die student moet vertroud wees met Python se  
             sintaksis en inkeping vir geneste strukture.

	Die student moet vertroud wees met geneste lyste ('n lys van lyste) asook die geheue model daarvan.	
	 Die student moet in staat wees om te identifiseer 
             wanneer die gekoppelde skikkings nodig is van die probleem stelling en
             probleem uiteensetting, asook om dit te implementeer.
                
\subsubsection {Tema 7: Plot en Grafieke} 
         Die student moet in staat wees om 2D grafieke te skep d.w.s. die student 
         moet vertroud wees met die verskillende grafiese aanbiedings van data op verskeie
	grafieke op dieselfde figuur, vertoon van roorsters, as-benoemings, legendes van meervoudige
	datastelle asook skalering van die assestelsel. Die student moet kennis neem van die \textit{array}
	voorwerp tipe wat geskik is met die gebruik van die \texttt {Numpy} module. Die student moet
	vertroud wees met die \textit{array} se geassosieerde operateurs (\texttt {+, -, *, **}). Die student
	moet vertroud wees met die skepping van 3D oppervalk en rooster grafieke.

\subsubsection {Tema 8: Datahantering} 
         Die student moet vertroud wees met die lees en skryf
         van \texttt {CSV} l\^eers. 
         Die student moet in staat wees om die data in 'n gegewe formaat te stoor.
	Die student moet ook verstaan ​​hoe om toegang tot data te verkry in Python
         gestoor deur LibreOffice en \textit {andersom}.

\subsubsection {Tema 9: Ho\"e Vlak Programmeering} 
                Die student moet Python se dokumentasie 
                sowel as aanlyn-dokumentasie kan verken. Die student moet in staat wees om die dokumentasie van 'n spesifieke funksie kan vind, lees en verstaan. Die student moet addisionele modules en funksies in Python kan
                toepas in probleemoplossing. Die student moet sy / haar kennis van die Python taal 	
    onafhanklik kan verbeter.

\subsubsection {Tema 10: Grafiese Gebruikerskoppelvlakke (GGK)} 
            Die student moet die Tkinter module kan gebruik vir die skep van 
            eenvoudige GGK. Die student moet eenvoudige 
            vormvoorwerpe kan skep (bv. knoppies, insette bokse, teks bokse, ens) en 
            op die GGK kan plaas. Die student moet ook
            vertroud wees met die hantering van die gebruikersgebeure (sleutelbord of die muis gebeure) 
            en in staat wees om hierdie gebeure en funksies te skakel om 
            'n probleem op te los deur 'n GGK te gebruik.


    \subsubsection{Tema 11: Sigblaaie}
        \paragraph{Formules en Berekeninge:}
            Die student moet in staat wees om probleme op te los deur gebruik 
            te maak van formules. Verder moet die student in staat wees om probleme
            op te los deur gebruik te maak van bekende funksies sowel as die
            funksie ``wizard'' vir onbekende funksies.  Laastens moet die student in staat
            wees om die reekse van selle te benoem en te gebruik in berekeninge.
            
    Die student moet in staat wees om filters te gebruik om data te filter
    en berekeninge te verrig op di\'{e} datastel.  Die student
    moet in staat wees om berekeninge te verrig met data wat verdeel is
    tussen verskillende sigblaaie.
    
        \paragraph{Sigblaai Inspekteur:}
    Studente moet in staat wees om die sigbladinspekteur te gebruik
    om ten einde hulself vertroud te maak met die afhanklikhede in 'n 
    onbekende sigblad.  Die student moet in staat wees om te kan omskakel 
    tussen formule- en waardesig.

        \paragraph{Sketse en Grafieke:}
            Die student moet in staat wees om grafieke te skep en te verander.
            Die student moet ook tendense in die data kan vind.          

        \paragraph{Line\^{e}re Programmering:}
            Die student moet in staat wees om line\^{e}re programmeringsprobleme
            op te los met behulp van LibreOffice Calc. Die student
            moet ook in staat wees om probleme op te los met ``integer'', ``real''
            en ``boolean'' veranderlikes.  Die student moet ook in staat wees om 
            verskeie beperkinge te akkommodeer.

        \paragraph{Nie-Line\^{e}re Oplosser:}
            Die student moet in staat wees om 'n een dimensionele nie-line\^{e}re 
            probleem op te los deur gebruik te maak van LibreOffice Calc se 
            ``goal seek'' funksie.

        \paragraph{Data ``Pilot'':}
    Die student moet in staat wees om verwante data uit te trek 
    deur gebruik te maak van die Data ``Pilot'' in LibreOffice Calc.

        \paragraph{Wat-as Situasie:}
            Die student moet in staat wees om 'n `wat-as' situasies op te los 
            deur gebruik te maak van die 'scenario'-funksie  in LibreOffice Calc.

        \paragraph{Visuele Formatering:}
            Die student moet in staat wees om die uitleg van 'n sigblad te verbeter
            deur gebruik te maak van kleure, beklemtooning (verhelder), grense en lettertipe
            eienskappe.  Die student moet in staat wees om 'conditional formatting' 
            van selle uit te voer.

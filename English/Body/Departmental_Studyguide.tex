\section{Departmental Study Guide} \label{sec:dep_study_guide}
    This study guide is a crucial part of the general study guide of the
    Department. In the study guide of the Department , information is given on
    the mission and vision of the department , general administration and
    regulations (professionalism and integrity, course related information and
    formal communication, workshop use and safety, plagiarism, class
    representative duties, sick test and sick exam guidelines, vacation work,
    appeal process and adjustment of marks, university regulations, frequently
    asked questions), ECSA outcomes and ECSA exit level outcomes, ECSA
    knowledge areas, CDIO, new curriculum and assessment of cognitive levels.
    It is expected that you are very familiar with the content of the
    Departmental Study Guide. It is available in
    \href{http://www.up.ac.za/media/shared/120/Noticeboard/departmentalstudyguide_eng_2016_kenedit26nov2015.zp76159.pdf}{English}
    and
    \href{http://www.up.ac.za/media/shared/120/Noticeboard/departementele_studiegids_afr_2016_kenedit2des2015.zp76157.pdf}{Afrikaans}
    on the
    \href{http://www.up.ac.za/en/mechanical-and-aeronautical-engineering/article/21692/noticeboard}{Department’s website}.

    \noindent
    \textbf{English:} \\
    \url{http://www.up.ac.za/media/shared/120/Noticeboard/departmentalstudyguide_eng_2016_kenedit26nov2015.zp76159.pdf} \\~\\
    \textbf{Afrikaans:} \\
    \url{http://www.up.ac.za/media/shared/120/Noticeboard/departementele_studiegids_afr_2016_kenedit2des2015.zp76157.pdf} \\~\\
    \textbf{Department Website.:} \\
    \url{http://www.up.ac.za/en/mechanical-and-aeronautical-engineering/article/21692/noticeboard} \\~\\

    \noindent
    \textbf{Take note of the \uline{specific instructions} in the above study guide on:}
    \begin{itemize}
        \item \textbf{Safety}
        \item \textbf{Plagiarism}
        \item \textbf{What to do if you were sick (very important)?}
        \item \textbf{Appeal process on the adjustment of marks}
    \end{itemize}

\newpage
\section{Application for Credit} \label{sec:credit_study_guide}
    \noindent
    Credit for MPR213 will only be granted if the student
    \begin{itemize}
        \item complies with the sipulations set out by the general rules and
            regulations of the University (refer to
            \url{http://www.up.ac.za/new-students-undergraduate/article/256899/general-regulations-and-rules})
        \item passes an exit level outcome credit assessment exam
        \item has not registered for MPR213 in years prior 2016.
    \end{itemize}

    \noindent
    In order to apply for credit for MPR213 students must
    \begin{itemize}
        \item be registered for MPR213 in 2016.
        \item complete the credit application form obtained from student
            admin (Eng I, floor 6)
        \item submit the completed credit application form along with a
            copy of your full academic record to Mr Page (Eng III 6-93) before
            5 February 2016 (17:00). Late applications will not be considered.
    \end{itemize}

    \noindent
    Credit assessment exam details:
    \begin{itemize}
        \item Date: 12 February 2016
        \item Time: 14:00 to 15:30 (90min)
        \item Venue: NW2 PhD Lab
        \item Scope:
        \begin{itemize}
            \item 30\% LibreOffice (spreadsheets) -- Multiple Choice
            \item 70\% Python (solving problems) -- Written
            \item Refer to the study guide and lecture notes for more
                information and topics covered.
        \end{itemize}
        \item You will have access to a computer during the test to use to
            solve the given problem, after which the answer (code) needs to be
            written down on the question paper itself.
        \item There is no "sick test" for this assessment exam. Students who
            miss this assessment exam will be required to complete the course.
        \item Any changes will be communicated via email.
    \end{itemize}


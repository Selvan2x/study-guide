\section{Opvoedkundige Benadering}
    %Die belangrikheid van programmeringsvaardighede is vinnig besig om net 
    %so belangrik te word vir ingenieurs as hulle vaardigheid in wiskunde.  Die 
    %rede vir dit is die feit dat rekenaars die sakrekenaars van die toekoms is.
    %Vir 'n ingenieur om effektief 'n rekenaar te gebruik, moet hy ten minste 
    %bietjie vaardigheid opbou in programmering.
Programmeringskennis en -vaardighede is belangrik vir
ingenieurs. Komplekse berekeninge en repiterende analises wat dae met
die hand kan neem kan binne sekondes opgelos word deur gebruik te maak
van 'n goed gestruktureerde program. Alhoewel die program self tyd
neem om gekodeer te word, sodra die program gekodeer is laat dit die
ingenieur toe om verskillende vrae te vra en verskillende ondersoeke
te loots wat andersins nie moontlik sou wees nie. Programmering
gekombineer met wiskunde het 'n voorvereiste geword vir basiese
ingenieurs geletterheid en probleem oplossing. Dit is belangrik om op
te merk dat die lewenssiklus van 'n program dieselfde is as die
ontwerp van 'n produk of proses.

Dit behoort duidelik te wees hoekom dit noodsaaklik is dat ingenieurs
'n mate van vaardigheid in programmering het. Die algemene doelwitte
vir hierdie module is om die student in staat te stel om:
    \begin{itemize}
    \item ingenieursprobleme op te los deur die ontwikkeling en ontfouting
      ('debugging') van basiese rekenaarprogramme.
    \item rekenaar data te verwerk en uit te voer as informasie of
      kennis (teks en visueel).
    \end{itemize}

    Element\^{e}re wiskundige konsepte, waarmee die studente reeds vertroud 
    moet wees, sal gebruik word om die basiese rekenaarlogika en 
    programmeringsbeginsels 
    te illustreer. Daar sal van die student verwag word om 
    wiskundige formules en/of beginsels te vertaal na 'n werkende 
    rekenaarprogram
    om ander probleme mee op te los.

    Die \textbf{opvoedkundige benadering} is dus \textbf{probleem
      gedrewe}, m.a.w.  programmeringsvaardighede word vekry deur die
    oplos van probleme. Die bemeestering van die tutoriale en huiswerk
    probleme is noodsaaklik om sukselvol te wees in hierdie
    module. Soortgelyk aan 'n natuurlike taal wat toegewyde tyd in
    gesprek benodig, vereis programmering toegewyde tyd agter die
    rekenaar om probleme op te los. \textit{Maak seker jy sit voor 'n
      rekenaar en los daagliks ingenieurs probleme op in Python}.
    
    Student geori\"{e}nteerde en ko\"{o}peratiewe studiemetodes sal
    toegepas word gedurende die lesings en tutoriaalsessies om ten
    einde die kernkonsepte van die kursus oor te dra.  Lesings sal
    spesifiek fokus om probleme in die klas op te los en die
    programmeringskonsepte tydens lesings te ontwikkel, ten einde jou
    instaat te stel om wiskunde en fisika probleme selfstandig mee op
    te los.

    Daar word van studente verwag om deel te neem aan
    klasbesprekings. Hierdie klasbesprekings skep die geleentheid om
    ervarings te deel en probleme op te los as deel van 'n span.  Dit
    sal waardevolle blootstelling gee aan wat algemeen in industrie
    voorkom. Probleem geori\"{e}nteerde sessies tydens die lesings sal
    die studente die geleentheid gee om moeilike konsepte te
    assimileer en te verstaan. Studente word aangemoedig om
    skootrekenaars na die lesings te bring om die kode saam met die
    dosent in die klas te ontwikkel.




    

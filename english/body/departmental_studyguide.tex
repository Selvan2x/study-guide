\section{Departmental Study Guide} \label{sec:dep_study_guide}
    This study guide is a crucial part of the general study guide of the
    Department. In the study guide of the Department , information is given on
    the mission and vision of the department , general administration and
    regulations (professionalism and integrity, course related information and
    formal communication, workshop use and safety, plagiarism, class
    representative duties, sick test and sick exam guidelines, vacation work,
    appeal process and adjustment of marks, university regulations, frequently
    asked questions), ECSA outcomes and ECSA exit level outcomes, ECSA
    knowledge areas, CDIO, new curriculum and assessment of cognitive levels.
    It is expected that you are very familiar with the content of the
    Departmental Study Guide. It is available in
    \href{http://www.up.ac.za/media/shared/120/Noticeboard/departmental-studyguide-eng-2016.zp77597.pdf}{English}
    and
    \href{http://www.up.ac.za/media/shared/120/Noticeboard/departementele-studiegids-afr-2016.zp77599.pdf}{Afrikaans}
    on the
    \href{http://www.up.ac.za/en/mechanical-and-aeronautical-engineering/article/21692/noticeboard}{Department’s website}.


    \noindent
    \textbf{English:} \\
    \url{http://www.up.ac.za/media/shared/120/Noticeboard/departmental-studyguide-eng-2016.zp77597.pdf} \\~\\
    \textbf{Afrikaans:} \\
    \url{http://www.up.ac.za/media/shared/120/Noticeboard/departementele-studiegids-afr-2016.zp77599.pdf} \\~\\
    \textbf{Department Website.:} \\
    \url{http://www.up.ac.za/en/mechanical-and-aeronautical-engineering/article/21692/noticeboard} \\~\\

    \noindent
    \textbf{Take note of the \uline{specific instructions} in the above study guide on:}
    \begin{itemize}
        \item \textbf{Safety}
        \item \textbf{Plagiarism}
        \item \textbf{What to do if you were sick (very important)?}
        \item \textbf{Appeal process on the adjustment of marks}
    \end{itemize}

\newpage
\section{Recognition of Modules} \label{sec:credit_study_guide}
    \noindent
    Credit recognition for MPR213, based on recognition of other modules, will
    only be granted if the student
    \begin{enumerate}
        \item complies with the stipulations set out by the general regulations
            and rules of the University (refer to
            \url{http://www.up.ac.za/media/shared/370/ZP_Files/general-regulations-and-rules-2015.zp42443.pdf},
        \item passes an exit level outcome based credit assessment exam (50\%+),
        \item has not written the credit assessment exam in the past (prior 2017).
    \end{enumerate}

    \noindent
    Entrance to the credit assessment exam requires that the student must
    \begin{enumerate}
        \item currently be registered for MPR213,
        \item complete the credit application form, obtained from student
            admin \\ (Eng I, floor 6),
        \item submit the completed credit application form along with a copy of
            your full academic record and study guides of passed modules to Mr
            Page (Eng III 6-93). The last date for handing in is before 17:00
            on ?? February 2017.
    \end{enumerate}

    \noindent
    Details regarding the credit assessment exam:
    \begin{itemize}
        \item Date: ?? February 2017
        \item Time: 14:00 to 15:30 (90min)
        \item Venue: NW2 PhD Lab
        \item Scope:
        \begin{itemize}
            \item 30\% LibreOffice (spreadsheets) -- Multiple Choice
            \item 70\% Python (solving problems) -- Written
            \item Aditional information is available in the class lecture notes
                on ClickUP.
        \end{itemize}
        \item You will have access to a computer with Python and LibreOffice
            during the exam to help solve the given problems.
        \item All the answers (code) needs to be written down on the answer
            book provided.
        \item Only the answer book will be handed in, no code will be
            stored eletronically.
        \item There is no aditional "sick test" for this assessment exam.
            Students who miss this assessment exam will be required to complete
            the course.
        \item Any changes will be communicated via email / ClickUP.
    \end{itemize}

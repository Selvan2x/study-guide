\section{Study Components}
    \subsection{Purpose of the module}
        {\bf Advanced spreadsheet applications:} \\
        Named ranges, referencing, linear programming, solving non-linear
        equations, fitting regression lines, interpolation, manipulating
        large data sets, extracting information from data sets. \\

        \noindent
        {\bf Basic structured programming:} \\
        Data container types, looping, branching, subroutines, iteration,
        reading and writing data files. Development, coding, and debugging
        of simple programs in a high level programming
        language. Programming principles are illustrated via mathematical
        and physics concepts such as limits, differentiation, integration,
        linear algebra and, simple motion. Structured programming by coding
        and using functions you wrote yourself as well as using functions
        available in packages. Basic graphical output (plotting) is also
        covered.

    \subsection{Module Structure}
        The structure of this module is shown in table \ref{tab:study_comp}.
        The total module hours (notional hours) include the contact time,
        as well as the estimated time to be allocated for self-study,
        preparation of assignments, tests and the examination. The mode of
        instruction is via lectures, tutorial classes and, assignment. \\

        \vspace{-1em}
        \begin{longtable}{|p{2.0cm}|p{7.6cm}|p{2.0cm}|p{1.8cm}|}
             \hline
             \multicolumn{1}{|p{2.0cm}|}{\textbf{Theme No.}} & %
             \multicolumn{1}{p{7.6cm}|}{\textbf{Topic}} &  %
             \multicolumn{1}{p{2.0cm}|}{\textbf{Notional Hours}} & %
             \multicolumn{1}{p{1.8cm}|}{\textbf{Contact Sessions}} \\
             \hline \endfirsthead

             \multicolumn{4}{l}%
             {\textbf{...continued from previous page}} \\
             \hline
             \multicolumn{1}{|p{2.0cm}|}{\textbf{Theme No.}} & %
             \multicolumn{1}{p{7.6cm}|}{\textbf{Topic}} &  %
             \multicolumn{1}{p{2.0cm}|}{\textbf{Notional Hours}} & %
             \multicolumn{1}{p{1.8cm}|}{\textbf{Contact Sessions}} \\
             \hline

             \hline \endhead
             \hline \endfoot
             \endlastfoot

             1  & Introduction                               &    &    \\
             \hline
             2  & Basic Programming                          & 12 & 4  \\
                & \qquad Using Python as a Calculator        &    &    \\
                & \qquad Names, Objects and Assignment       &    &    \\
                & \qquad Import and Use Modules              &    &    \\
                & \qquad Python Namespace                    &    &    \\
             \hline
             3  & Basic Functions                            & 12 & 4  \\
                & \qquad User-defined Functions              &    &    \\
                & \qquad Python Namespace and Scoping        &    &    \\
                & \qquad Basic Numpy Arrays                  &    &    \\
             \hline
             4  & Control Statements                         & 34 & 10 \\
                & \qquad Iterator-Based Looping (For Loop)   &    &    \\
                & \qquad Conditional Looping (While Loop)    &    &    \\
                & \qquad Branching (If, Elif, Else)          &    &    \\
             \hline
             5  & Solving Problems                           & 35 & 8  \\
                & \qquad Breaking Down Problem Complexity    &    &    \\
                & \qquad Use of Functions (Simplification)   &    &    \\
                & \qquad Nested Statements                   &    &    \\
             \hline
             6  & Plotting and Graphs                        & 12 & 4  \\
                & \qquad 2D Graphs                           &    &    \\
                & \qquad Graph Annotations                   &    &    \\
                & \qquad 3D Graphs                           &    &    \\
             \hline
             7  & Reading and Writing Data                   &  6 & 2  \\
                & \qquad Data Sharing between Application    &    &    \\
             \hline
             8  & High Level Programming                     & 14 & 4  \\
                & \qquad Additional Modules                  &    &    \\
                & \qquad Advanced Built-In Functions         &    &    \\
             \hline
             9  & Graphical User Interfaces (GUI's)          &  5 & 2  \\
                & \qquad Tkinter Module                      &    &    \\
                & \qquad Forms and Form Objects              &    &    \\
                & \qquad Mouse / Keyboard Events             &    &    \\
             \hline
             10 & Spreadsheets                               & 30 & 8  \\
                & \qquad Formulas and Calculations           &    &    \\
                & \qquad Spreadsheet Detective               &    &    \\
                & \qquad Plotting and Graphs                 &    &    \\
                & \qquad Linear Programming                  &    &    \\
                & \qquad Data Lookups and Pivots             &    &    \\
             \hline
                & {\bf TOTAL}                                &160 & 46 \\
             \hline

             \caption[Module Structure]{Module Structure} \label{tab:study_comp} \\
        \end{longtable}

    \subsection{Lecture Plan}
        This outlines the lecture plan for the semester. Students should use
        this plan to make sure that they do not fall behind the class
        lectures and/or tutorials. There are a total of 48 lectures and 12
        tutorial sessions over the 14 weeks of the semester.

        Table \ref{tab:lec_plan} lists the number of lectures\footnote{Number
        of English/Afrikaans lectures per week} per week and which study themes
        will be covered in which weeks of the semester. Each week has a
        corresponding tutorial which should be completed during the specific
        week in the tutorial sessions.

        \begin{table}[!h]
            \begin{center}
             \begin{tabular}{|l|l|l|l|l|}
                 \hline
                 {\bf Week} & {\bf Lectures} & {\bf Dates} & %
                 {\bf Study Theme} & {\bf Tutorial} \\
                 \hline
                 1  & 4/4   &  6 –- 10 Feb      & 1, 2  & None \\
                 2  & 4/4   & 13 –- 17 Feb      & 3     & T1 \\
                 3  & 4/4   & 20 –- 24 Feb      & 4     & T2 \\
                 4  & 4/4   & 27 –-  3 Mar      & 4     & T3 \\
                 5  & 4/4   &  6 –- 10 Mar      & 4, 5  & T4 \\
                    &       & == TESTWEEK 1 == &       & \\
                 6  & 2/2   & 22 -- 24 Mar      & 5     & T5 \\
                 7  & 4/4   & 27 -- 31 Mar      & 5     & T6 \\
                 8  & 4/4   &  3 --  7 Apr      & 6     & T7 \\
                    &       & ==== RECESS ==== &       & \\
                 9  & 2/0   & 19 -- 21 Apr      & 10    & T8 \\
                 10 & 2/4   & 24 -- 26 Apr      & 10    & T8 \\
                 11 & 4/4   &  2 --  5 May      & 10    & T9 \\
                    &       & == TESTWEEK 2 == &       & \\
                 12 & 4/4   & 15 -- 19 May      & 7, 8  & T10 \\
                 13 & 4/4   & 22 -- 26 May      & 8, 9  & T11 \\
                 14 & 2/2   & 29 -- 30 May      & Recap & T12 \\
                    &       & ===== EXAM ===== &       & \\
                 \hline
             \end{tabular}
             \caption{Lecture Plan} \label{tab:lec_plan}
            \end{center}
        \end{table}

        Notional hours include contact time as well as the estimated time
        necessary for preparation for tests, and exams. Contact sessions
        indicate the regular lectures. The number of contact sessions per
        chapter is tentative and may change depending on the progress during
        lectures.

        Please note that some subsections in a chapter of the textbook will not
        be covered during the lectures, whereas other subsections may be given
        as self-study. The lecturer will provide information about the
        sub-sections that will not be covered for test- and exam purposes.

        Selected exercise problems, from the study notes, will be covered
        during the tutorial sessions. The selected chapters and problems (from
        the study notes) that will be covered in each tutorial session will be
        added to {\it ClickUP}.

    \subsection{Fundamental Concepts}
        The following concepts need to be mastered in order to pass the course.
        If any of the following concepts are not mastered, the student will
        fail the course:
        \begin{itemize}
            \item Basic objects (e.g. int vs float)
            \item Importing modules and proper usage
            \item Functions (able to properly create a new function, understand
                how objects are passed to and from functions, be able to
                properly use any created function)
            \item For loop (proper usage and understanding when and why to use
                a for loop)
            \item Iterating through lists and growing lists (single lists, not
                nested lists)
            \item While loop (proper usage, understanding termination
                conditions and understanding when and why to use a while loop)
            \item If statements (proper usage, understanding program flow and
                control and understanding when and why to use an if, elif or
                else statement)
            \item Reading and writing of {\it CSV} files
            \item Plotting (be able to create a simple 2D plot with proper
                annotations)
            \item Data processing using spreadsheets
        \end{itemize}

    \subsection{Study Theme Descriptions}
        \subsubsection{Theme 1: Introduction}
            The introduction allows the student to obtain a general overview on
            the roles and responsibilities of both the student and the lecturer
            as well as an overview on what computer programming is. This
            section will not be explicitly tested. It is important to note that
            an understanding of computer architecture and flow diagrams will
            make the following sections more accessible.

        \subsubsection{Theme 2: Basic Programming}
            \paragraph{Using Python as a Calculator:}
                The student has to ensure that he/she is comfortable with the
                \emph{Jupyter} environment and must be able to use Python to do
                simple mathematical calculations. It is the student's
                responsibility to spend enough time using these environments
                throughout the semester. The student must be able to use
                Python's built-in functions.

            \paragraph{Names, Objects and Assignment:}
                The student has to be familiar with the guidelines in the class
                notes on names and objects e.g. objects are created in memory
                and a name is assigned to that object. The student must also be
                familiar with the memory model of Python. The student must also
                know the difference between the object types e.g. int, float,
                str, bool as well as to know how to check the type of an object
                using \texttt{type()}.

        \subsubsection{Theme 3: Functions}
            The student must be familiar with functions and code structure.
            The student must be able to properly create and use functions with
            multiple inputs and outputs.  The student must be able to rewrite
            an existing computer program such that it makes use of functions.
            The student has to be familiar with local namespace and understand
            ``name scope'' within a function.  The student must understand any
            object can be be passed into a function or returned from a
            function. This includes another function and is particularly
            important for some of the built-in functions.

        \subsubsection{Theme 4: Control Statements}
            \paragraph{For Loop:}
                The student must understand that the \textit{for loop} iterates
                through iterators (e.g. the elements of a list). The student
                has to be able to identify when and why to use \textit{for
                loop(s)} from a given problem statement. The student has to be
                familiar with Python's syntax and indentation and be
                comfortable to implement \textit{for loop(s)} in a program to
                perform a given task.

            \paragraph{While Loop:}
                The student must understand conditional statements and how to
                combine them using \texttt{and} and \texttt{or} operators. The
                while continues to iterate as long as the condition evaluated
                is \texttt{true} or \texttt{1}. The student must understand
                that should the condition not update during a \textit{while
                loop} iteration, the program is stuck in an infinite loop. The
                student has to be able to identify when and why to use while
                loop(s) from a given problem statement. The student has to be
                familiar with Python's syntax and indentation and be able to
                implement while loop(s) in a program to perform a given task.
                The student also has to be able to determine the correct
                \textit{while loop} termination conditions in order to perform
                a given task.

            \paragraph{Basic List Generation:}
                The student must be able to generate simple list objects, both
                manually and by using the range function. The student also has
                to be familiar with the various list operators \texttt{(+, *)},
                functions (\texttt{sorted}) and in-place functions (methods)
                (\texttt{append, insert, index, sort, reverse}). The student
                must be able to use, read and understand Python's built-in help
                to obtain information on the various functions and in-place
                functions.

            \paragraph{Branching:}
                The student must understand conditional statements and how to
                combine them using \texttt{and} and \texttt{or} operators. The
                student has to be able to identify the usage of branches (when
                and why to use an if, elif or else statement) from a problem
                statement. The student has to be familiar with Python's syntax
                and indentation and be able to implement branches in a program
                to perform a given task. The student needs to understand how
                each of the if, elif and else statements affects the program
                flow and control. The student also has to be able to determine
                the correct conditions in order to perform a given task.

        \subsubsection{Theme 5: Solving Problems}
            The student must be able to evaluate and break down a complex
            problem into logical control statements (for, while, if). The
            student must be able to identify which nested statements are needed
            from the problem statement and breakdown and then be able to
            implement the nested statements in a program to solve the given
            problem. The student has to be familiar with Python's syntax and
            indentation for nested statements.

            The student must also be familiar with nested lists (list of lists)
            and the memory model thereof. The student must be able to identify
            when nested lists are needed from the problem statement and
            breakdown and be able to implement nested lists.

        \subsubsection{Theme 6: Plotting and Graphs}
            The student must be able to create 2D plots i.e. the student must
            be familiar with different presentations of data, presenting
            multiple graphs on the same figure, displaying grids, labelling of
            the axis, naming of figures, using legends, and scaling axis
            systems. The student must take note of arrays which are an object
            type of the \texttt{numpy} module and understand their associated
            operators (\texttt{+, -, *, **}). The student must be familiar
            with creating 3D figures. The student has to be familiar with
            creating surface and mesh plots.

        \subsubsection{Theme 7: Reading and Writing Data}
            The student must be familiar with reading and writing to and from
            \texttt{CSV} files. The student must be able to store the data in
            a given format. The student must also understand how to access
            data, stored by Python, in LibreOffice and \textit{vice versa}.

        \subsubsection{Theme 8: High Level Programming}
            The student must be able to navigate Python's documentation as well
            as online documentation and be able to find, read and understand
            the documentation of the required functions in order to solve
            specific problems. The student should therefore be able to
            independently increase his/her knowledge of the Python programming
            language.

            The student must then be able use these additional modules and
            functions in Python in order to solve a specific problem.

        \subsubsection{Theme 9: Graphical User Interfaces (GUI's)}
            The student has to be familiar with the Tkinter module for creating
            simple GUI's. The student has to be familiar with creating simple
            form objects (e.g. buttons, input boxes, text boxes, etc.) and be
            able to create these form objects on a GUI. The student also has to
            be familiar with handling user events (keyboard or mouse events)
            and be able to link these events to functions in order to solve a
            given problem through using a GUI.

        \subsubsection{Theme 10: Spreadsheets}
            \paragraph{Formulas and Calculations:}
                The student must be able to solve problems using formulas. The
                student must be able to solve problems using functions that
                he/she are familiar with, as well as use the function wizard
                for unknown functions. The student must be able to name ranges
                of cells and use them in calculations.

                The student must be able to use filters to filter data and
                perform calculations on the filtered dataset. The student must
                be able to perform calculations on data that is split over
                multiple spreadsheets.

            \paragraph{Spreadsheet Detective:}
                The student must be able to use the spreadsheet detective to
                familiarise themselves with the dependencies in an unknown
                spreadsheet. The student must be able to switch between formula
                and value view.

            \paragraph{Plotting and Graphs:}
                The student must be able to create and customise charts. The
                student must be able to find trends in data.

            \paragraph{Non-Linear Solver:}
                The student must be able to solve non-linear one dimensional
                problems using goal seek in LibreOffice Calc.

            \paragraph{Linear Programming:}
                The student must be able to solve linear programming problems
                using LibreOffice Calc. The student must be able to solve
                problems with mixed integer, real and, boolean variables. The
                student must be able to accommodate multiple constraints.

            \paragraph{Data Lookups and Pivots:}
                The student must be able to extract relational data using the
                Pivot Tables in LibreOffice Calc.

            \paragraph{Visual Formatting:}
                The student must be able to improve the appearance of a
                worksheet by using colours, highlighting, borders, and font
                properties. The student must be able to perform conditional
                formatting of cells.

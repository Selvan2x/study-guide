\section{Studie Materiaal en Sagteware}
    \subsection{Voorgeskrewe Handboek}
        Daar is geen voorgeskrewe handboek vir hierdie kursus nie. Die volgende 
        studienotas is by UP ontwikkel en sal beskikbaar wees op Click-UP:
        
        ``\underline{Introduction to programming for engineers using
          Python}'' deur Logan G. Page, Daniel N. Wilke en Schalk Kok.
    
    \subsection{Komplement\^{e}re Bronne}
        %Die volgende aanvullende bron van notas sal ook beskikbaar gestel
        %word op \textit{ClickUP}:
        Aanvulende notas sal ook beskikbaar wees op \textit{ClickUP}:
            
        ``\underline{Python for Computational Science and Engineering}'' deur 
        Hans Fangohr
        
        Talle aanvulllende bronne vir hierdie kursus word bespreek in die
        bogenoemde studienotas.  Die besprekingsforum en wiki-bladsy
        op die \textit{ClickUP} stelsel sal ook gebruik word vir addisionele
        verduidelikings en voorbeelde oor verskeie kursusonderwerpe. 
        
        Notas van additionele studiemateriaal (wat nie gedek word in die voorgeskrewe
        notas nie) sal elektronies beskikbaar wees op \textit{ClickUP}. Hierdie 
        notas sal ook deel wees van die sillabus.  Let wel, hierdie additionele notas 
        dek nie \'{a}l die werk wat bespreek word gedurende klas nie en studente
        moet asseblief hulle eie aanvulllende notas maak gedurende klastye.
        
        Uitgewerkte oplossings wat gedurende die lesings in detail gedek word, sal nie
        later beskikbaar gemaak word op \textit{ClickUP} nie.
        

    \subsection{Vereiste Sagteware}
	In hierdie kursus maak ons gebruik van vrylik beskikbare sagteware.
	Dit impliseer dat studente die sagteware kan gebruik en kopieer sonder
	enige wettige beparkings.

	Die programmeringspakket wat gebruik word in hierdie kursus is Python.
	Python is 'n ho\"{e}-vlak programmeringstaal wat gerbuik word in baie
	dissiplines reg oor die w\^{e}reld. Python is 'n goed geskikte programmeringstaal 
	vir ingenieurs omdat dit maklik is om te leer.
	Python is ook goed gedokumenteer, relatief vinnig en goed geskik vir numeriese 
	en wetenskaplike rekenaarvaardighede.
        
        Jy kan Python aflaai ({\tt Python(x,y)-2.7.5.2.exe}) by:
        \begin{itemize}
            \item \textit{ClickUP} (aanbeveel) of
            \item \url{http://code.google.com/p/pythonxy/wiki/Downloads}
        \end{itemize}
        (Sien die bogenoemde studienotas vir installeeringsinstruksies)

	In hierdie kursus word LibreOffice Calc gebruik vir sigblaaie (spreadsheets).
	LibreOffice Calc is grootendeels soortgelyk aan Microsoft Excel, met die
	uitsondering dat dit afgelaai en gebruik kan word sonder wettige
	beperkings.
        
        Jy kan LibreOffice aflaai:
        ({\tt  LibreOffice\_4.1.4\_Win\_x86.msi}) van:
        \begin{itemize}
            \item \textit{ClickUP} (aanbeveel) of
            \item \url{http://www.libreoffice.org/download/?type=win-x86&
                       version=4.1.4&lang=en-ZA}
        \end{itemize}
        (Sien \url{http://www.libreoffice.org/get-help/installation/windows/} vir 
        installeeringsinstruksies)

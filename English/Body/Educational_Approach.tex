\section{Educational Approach}
%    The importance of programming skills is rapidly becoming as
%    important to engineers as their skill in mathematics. The
%    reason for this being the fact that a computer is tomorrows
%    calculator. For an engineer to effectively make use of this fact
%    it is required that he has at least some proficiency in
%    programming.
    
    Programming knowledge and skills are important for engineers. Complex calculations 
    and analysis that would take days if done by hand can take only seconds with a well 
    structured program. It is for this reason that engineers obtain at least some 
    proficiency in programming.

    The general objectives of this module are:
    \begin{itemize}
        \item To enable the students to solve engineering problems by developing, 
              debugging, and running basic computer programs.
        \item To enable the students to process computer data and output (text and
              visual) information or knowledge.
    \end{itemize}

    Elementary mathematical concepts, which the students should
    already be familiar with, will be used to illustrate basic
    computer logic and programming principles. The student will be
    required to 'translate' mathematical formulas and/or principles
    into a working computer program in order to solve some problems.

    The educational approach is problem driven, in other words
    programming skills are acquired by solving problems. The mastering
    of tutorial and homework problems is therefore essential to be successful in 
    this module.
    
    Student-orientated and cooperative study methods will be applied during the
    lectures and tutorial sessions in order to establish the core concepts of 
    the course. Students are expected to participate in discussions in class since
    this creates an opportunity to share experiences and solve problems in a team 
    orientated environment. This will mimic what generally occurs in industry.
    Problem solving sessions during lectures will provide students with 
    opportunities to assimilate and understand difficult concepts.

    \subsection{Departmental Study Guide}
        This study guide is a crucial part of the general Departmental study guide.
        In the Departmental study guide, information is given
        on the mission and vision of the Department, general administration and
        regulations (professionalism and integrity, course related information
        and formal communication, workshop use and safety, plagiarism, class
        representative duties, sick test and sick exam guidelines, vacation
        work, university regulations, frequently asked questions), ECSA
        outcomes and ECSA exit level outcomes, ECSA knowledge area, CDIO,
        new curriculum and assessment of cognitive levels. It is expected that
        you are very familiar with the content of the Departmental Study Guide.
        It is available in English and Afrikaans on the
        \href{http://web.up.ac.za/default.asp?ipkCategoryID=11426&subid=11426&%
              ipklookid=7}{Department’s website.\footnote{%
              \url{http://web.up.ac.za/default.asp?ipkCategoryID=11426&%
                   subid=11426&ipklookid=7}}}
                   
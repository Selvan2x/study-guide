\section{Study Material and Software}
    \subsection{Prescribed Textbook}
        There is no prescribed textbook for this course, however the following 
        study notes have been developed at UP and will be available on Click-UP:
        
        ``\underline{Introduction to programming for engineers using Python}'' by
        Logan G. Page, Daniel N. Wilke and Schalk Kok.
    
    \subsection{Complementary Sources}
        The following complementary source of notes will also be made available
        on Click-UP:
            
        ``\underline{Python for Computational Science and Engineering}'' by 
        Hans Fangohr
        
        Numerous other complementary sources for this course are mentioned in the 
        study notes above. The discussion forum and wiki page on the Click-UP 
        system will also be utilised for additional explanations and examples 
        on various topics covered in this course.
        
        Any notes on study material not covered in the study notes will be made 
        available in electronic format on Click-Up. These additional notes will also 
        be part of the syllabus. Lecture slides will be made available on 
        Click-Up. Please note that these lecture slides do not cover all the
        work discussed in class and students should take down their own 
        supplementary notes during lectures. 
        
        Problem solutions covered in detail during the lectures will not be 
        made available again at a later stage.


    \subsection{Required Software}
        In this course we make use of open source software. This implies
        that students can freely copy and use the software legally
        without any restrictions.

        The programming package used in this course is Python. Python is
        high level programming language used in many disciplines around the 
        world. Python is a well suited programming language for engineers as 
        it is easy to learn, well supported and documented, relatively fast, 
        and well suited for numerical and scientific computing.
        
        You can download Python ({\tt Python(x,y)-2.7.3.1.exe}) from:
        \begin{itemize}
            \item Click-UP (recommended) or
            \item \url{http://code.google.com/p/pythonxy/wiki/Downloads}
        \end{itemize}
        (See the study notes, mentioned above, for installation instructions)

        LibreOffice Calc will be used for spreadsheets in this course. It is 
        in many ways very similar to Microsoft Excel with the exception 
        that it can be downloaded and used without any restrictions.
        
        You can download LibreOffice 
        ({\tt LibO\_3.6.4\_Win\_x86\_install\_multi.msi}) from:
        \begin{itemize}
            \item Click-UP (recommended) or
            \item \url{http://www.libreoffice.org/download/?type=win-x86&%
                       lang=en-ZA&version=3.6.4}
        \end{itemize}
        (See \url{http://www.libreoffice.org/get-help/installation/windows/} for 
         installation instructions)

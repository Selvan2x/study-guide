\section{Study Components}
    \subsection{Purpose of the module}
    {\bf Advanced spreadsheet applications:} \\
    Named ranges, referencing, linear programming, solving non-linear
    equations, fitting regression lines, interpolation, manipulating
    large data sets, extracting information from data sets. \\ \\
    {\bf Basic structured programming:} \\
    Data container types, Looping, branching, subroutines, iteration,
    reading and writing data files. Development, coding and debugging
    of simple programs in a high level programming
    language. Programming principles are illustrated via mathematical
    and physics concepts such as limits, differentiation, integration,
    linear algebra and simple motion. Structured programming by coding
    and using functions you wrote yourself as well as using functions
    available in packages. Basic graphical output (plotting) is also
    covered.

    \subsection{Module Structure}
        The structure of this module is shown in table \ref{tab:study_comp}.
        The total module hours (notional hours) include the contact time,
        as well as the estimated time to be allocated for self-study,
        preparation of assignments, tests and the examination. The mode of
        instruction is via lectures, tutorial class and assignment. \\

        \begin{longtable}{|p{1.4cm}|p{8.4cm}|p{2.2cm}|p{1.8cm}|}
             \hline
             \multicolumn{1}{|p{1.4cm}|}{\textbf{Theme No.}} & %
             \multicolumn{1}{p{8.4cm}|}{\textbf{Topic}} &  %
             \multicolumn{1}{p{2.2cm}|}{\textbf{Notional Hours}} & %
             \multicolumn{1}{p{1.8cm}|}{\textbf{Contact Sessions}} \\
             \hline \endfirsthead

             \multicolumn{4}{l}%
             {\textbf{...continued from previous page}} \\
             \hline
             \multicolumn{1}{|p{1.4cm}|}{\textbf{Theme No.}} & %
             \multicolumn{1}{p{8.4cm}|}{\textbf{Topic}} &  %
             \multicolumn{1}{p{2.2cm}|}{\textbf{Notional Hours}} & %
             \multicolumn{1}{p{1.8cm}|}{\textbf{Contact Sessions}} \\
             \hline

             \hline \endhead
             \hline \endfoot
             \endlastfoot

             1  & Introduction to Computers and Programming  &    &    \\
             \hline
             2  & Basic Programming                          & 12 & 4  \\
                & \qquad Using Python as a Calculator        &    &    \\
                & \qquad Names, Objects and Assignment       &    &    \\
                & \qquad Swapping Objects                    &    &    \\
                & \qquad Basic Lists                         &    &    \\
             \hline
             3  & Control Statements                         & 56 & 12 \\
                & \qquad Iterator-Based Looping (For Loop)   &    &    \\
                & \qquad Conditional Looping (While Loop)    &    &    \\
                & \qquad Branching (If, Elif, Else)          &    &    \\
             \hline
             4  & Lists Continued                            &  8 & 2  \\
                & \qquad Iterating Through Lists             &    &    \\
                & \qquad Growing Lists + Loops               &    &    \\
             \hline
             5  & Structured Programming (Functions)         &  8 & 2  \\
                & \qquad User-defined Functions              &    &    \\
                & \qquad Code Structure (modules vs scripts) &    &    \\
                & \qquad Local Namespace                     &    &    \\
             \hline
             6  & Nested Structures                          & 20 & 4  \\
                & \qquad Nested Control Statements           &    &    \\
                & \qquad Nested Iterators                    &    &    \\
                & \qquad Use of Functions (Simplification)   &    &    \\
             \hline
             7  & Plotting and Graphs                        & 16 & 4  \\
                & \qquad Numpy Objects and Functions         &    &    \\
                & \qquad 2D Graphs                           &    &    \\
                & \qquad Graph Annotation                    &    &    \\
                & \qquad 3D Graphs                           &    &    \\
             \hline
             8  & Data Handling                              & 12 & 3  \\
                & \qquad Reading and Writing Data            &    &    \\
                & \qquad Data Sharing between Application    &    &    \\
             \hline
             9  & High Level Programming                     & 10 & 3  \\
                & \qquad Additional Modules                  &    &    \\
                & \qquad Advanced Built-In Functions         &    &    \\
             \hline
             10 & Graphical User Interfaces (GUI's)          &  6 & 2  \\
                & \qquad TKinter Module                      &    &    \\
                & \qquad Forms and Form Objects              &    &    \\
                & \qquad Mouse / Keyboard Events             &    &    \\
             \hline
             11 & Spreadsheets                               & 32 & 8  \\
                & \qquad Formulas and Calculations           &    &    \\
                & \qquad Spreadsheet Detective               &    &    \\
                & \qquad Plotting and Graphs                 &    &    \\
                & \qquad Linear Programming                  &    &    \\
                & \qquad Non-Linear Solver problems          &    &    \\
                & \qquad Data Pilot                          &    &    \\
                & \qquad What-if Scenarios                   &    &    \\
                & \qquad Visual Formatting                   &    &    \\
             \hline
                & {\bf TOTAL}                                &180 & 44 \\
             \hline

             \caption[Module Structure]{Module Structure} \label{tab:study_comp} \\
        \end{longtable}

    \subsection{Lecture Plan}
        This outlines the lecture plan for the semester. Students should use
        this plan to make sure that they do not fall behind the class
        lectures and/or tutorials. There are a total of 48 lectures and 12
        tutorial sessions over the 13 weeks of the semester.

        Table \ref{tab:lec_plan} lists which study themes will be covered
        in which weeks of the semester. Each week has a corresponding tutorial
        which should be completed during the specific week in the tutorial
        sessions.

        \begin{table}[!h]
            \begin{center}
             \begin{tabular}{|l|l|l|l|l|}
                 \hline
                 {\bf Week} & {\bf Lectures} & {\bf Dates} & %
                 {\bf Study Theme} & {\bf Tutorial} \\
                 \hline
                 1  & 4     &  1 –-  5 Feb     & 1, 2  & None \\
                 2  & 4     &  8 –- 12 Feb     & 3     & T1 \\
                 3  & 4     & 15 –- 19 Feb     & 3     & T2 \\
                 4  & 4     & 22 –- 26 Feb     & 3     & T3 \\
                    &       & == TESTWEEK 1 == &       & \\
                 5  & 4     &  7 –- 11 Mar     & 4, 5  & T4 \\
                 6  & 4     & 14 -- 18 Mar     & 6     & T5 \\
                    &       & ==== RECESS ==== &       & \\
                 7  & 4     &  4 --  8 Apr     & 11    & T6 \\
                 8  & 4     & 11 -- 15 Apr     & 11    & T7 \\
                    &       & == TESTWEEK 2 == &       & \\
                 9  & 2     & 25 -- 29 Apr     & 7     & T8 \\
                 10 & 4     &  3 --  6 May     & 7, 8  & T9 \\
                 11 & 4     &  9 -- 13 May     & 8, 9  & T10 \\
                 12 & 4     & 16 -- 20 May     & 10, Recap & T11 \\
                 13 & 2 / 4 & 23 -- 25 May     & Recap & T12 \\
                    &       & ===== EXAM ===== &       & \\
                 \hline
             \end{tabular}
             \caption{Lecture Plan} \label{tab:lec_plan}
            \end{center}
        \end{table}

        Notational hours include contact time, as well as the estimated time
        necessary for preparation for tests and exams. Contact sessions
        indicate the regular lectures. The number of contact sessions per
        chapter is tentative. It may change depending on the progress during
        lectures.

        Please note that some subsections in a chapter of the textbook will not
        be covered during the lectures, whereas other subsections may be given
        as self-study. The lecturer will provide information about the
        sub-sections that will not be covered for test- and exam purposes.

        Selected exercise problems, from the study notes, will be covered
        during the tutorial sessions. The selected chapters and problems (from
        the study notes) that will be covered in each tutorial session will be
        added to {\it ClickUP}.

    \subsection{Fundamental Concepts}
        The following concepts need to be mastered in order to pass the course.
        If any of the following concepts are not mastered, the student will
        fail the course:
        \begin{itemize}
            \item Basic Objects (e.g. int vs float)
            \item Importing modules and proper usage
            \item For Loop (Proper usage and understanding when and why to use
                a for loop)
            \item Iterating through Lists and growing lists (single lists, not
                nested lists)
            \item While Loop (Proper usage, understanding termination
                conditions and understanding when and why to use a while loop)
            \item If Statements (Proper usage, understanding program flow and
                control and understanding when and why to use an if, elif or
                else statement)
            \item Functions (able to properly create a new function, understand
                how objects are passed to and from functions, be able to
                properly use any created function)
            \item File reading and writing of {\it CSV} files
            \item Plotting (be able to create a simple 2D plot with proper
                annotations)
            \item Data processing using spreadsheets
        \end{itemize}

    \subsection{Study Theme Descriptions}
        \subsubsection{Theme 1: Introduction to Computers and Programming}
        The introduction allows the student to obtain a general
        overview on the roles and responsibilities of both the student
        and the lecturer as well as an overview on what computer
        programming is.  This section will not be explicitly
        tested. It is important to note that an understanding of
        computer architecture and flow diagrams will make the
        following sections more accessible.

        \subsubsection{Theme 2: Basic Programming}
            \paragraph{Using Python as a Calculator:}
            The student has to ensure that he/she is comfortable with
            the \emph{IPython Console} and \emph{Spyder} environments
            and must be able to use Python to do simple mathematical
            calculations.  It is the student's responsibility to spend
            enough time using these environments throughout the
            semester.  The student must to use Python's built-in
            \texttt{math} module and access the functions stored in
            it. The student must be able to use Python's built-in help
            to obtain information on the various functions.

            \paragraph{Names, Objects and Assignment:}
            The student has to be familiar with the guidelines in the
            class notes on names and objects e.g. objects are created
            in memory and a name is assigned to that object. The
            student must also be familiar with the memory model of
            Python. The student must also know the difference between
            the object types e.g. int, float, str, bool as well as to
            know how to check the type of an object using
            \texttt{type()}.

            \paragraph{Swapping Objects:}
            The student must be able to swap the names bound to two objects without
            losing an object. The student also has to be
            familiar with the memory model on how the name binding
            to the objects are swapped.

            \paragraph{Basic List Generation:}
            The student must be able to generate simple list objects,
            both manually and by using the range function. The student
            also has to be familiar with the various list operators
            \texttt{(+,*)}, functions (\texttt{sorted}) and in-place
            functions (methods)
            \texttt{append,insert,index,sort,reverse}).  The student
            must be able to use, read and understand Python's built-in
            help to obtain information on the various functions and
            in-place functions.

        \subsubsection{Theme 3: Control Statements}
            \paragraph{For Loop:}
            The student must understand that the \textit{for loop}
            iterates through iterators (e.g. the elements of a
            list). The student has to be able to identify when and why
            to use \textit{for loop(s)} from a given problem
            statement. The student has to be familiar with Python's
            syntax and indentation and be comfortable to implement
            \textit{for loop(s)} in a program to perform a given task.

            \paragraph{While Loop:}
            The student must understand conditional statements and how
            to combine them using \texttt{and} and \texttt{or}
            operators. The while continues to iterate as long as the
            condition evaluated is \texttt{true} or \texttt{1}. The
            student must understand that should the condition not
            update during a \textit{while loop} iteration, the program
            is stuck in an infinite loop (\texttt{CTRL-C} breaks the
            program from the loop).  The student has to be able to
            identify when and why to use while loop/s from a given
            problem statement. The student has to be familiar with
            Python's syntax and indentation and be able to implement
            while loop/s in a program to perform a given task.  The
            student also has to be able to determine the correct while
            loop termination conditions in order to perform a given
            task.

            \paragraph{Branching:}
            The student must understand conditional statements and how
            to combine them using \texttt{and} and \texttt{or}
            operators. The student has to be able to identify the
            usage of branches (when and why to use an if, elif or else
            statement) from a problem statement. The student has to be
            familiar with Python's syntax and indentation and be able
            to implement branches in a program to perform a given
            task. The student needs to understand how each of the if,
            elif and else statements affects the program flow and
            control. The student also has to be able to determine the
            correct conditions in order to perform a given task.

        \subsubsection{Theme 4: Iterators (Lists)}
        The student has to understand the principles of lists as well
        as the Python syntax thereof. The student has to understand
        that a list is a multi-object container. The student must be
        able to create and grow lists as well as access and modify
        elements of a list.  The student must be able to identify and
        be able to implement lists as well as access and manipulate
        the objects stored in a list. The student must be able to
        rewrite an existing computer program that uses multiple single
        element container objects such that it uses lists. The student
        must also be able to identify when and why lists need to be
        grown and be able to implement this.

        \subsubsection{Theme 5: Structured Programming (Functions)}
        The student must be familiar with functions and code
        structure.  The student must be able to properly create and
        use functions with multiple inputs and outputs. The student
        must be able to rewrite an existing computer program such that
        it makes use of functions.  The student has to be familiar
        with local namespace and understand "name scope" within a
        function, module and script. The student must understand any
        object can be be passed into a function or returned from a
        function. This includes another function and is particularly
        important for some of the built-in functions.

        \subsubsection{Theme 6: Nested Structures}
            The student must be able to evaluate and break down a complex
            problem into logical control statements (for, while, if). The
            student must be able to identify which nested statements are needed
            from the problem statement and breakdown and then be able to
            implement the nested statements in a program to solve the given
            problem. The student has to be familiar with Python's syntax and
            indentation for nested statements.

            The student must also be familiar with nested lists (list of lists)
            and the memory model thereof. The student must be able to identify
            when nested lists are needed from the problem statement and
            breakdown and be able to implement nested lists.

        \subsubsection{Theme 7: Plotting and Graphs}
        The student must be able to create 2D plots i.e. the student
        must be familiar with different presentations of data,
        presenting multiple graphs on the same figure, displaying
        grids, labelling of the axis, naming of figures, using legends
        and scaling axis systems. The student must take note of
        array's which is an object type of the \texttt{numpy} module
        and understand their associated operators (\texttt{+,-,*,**}).
        The student must be familiar with creating 3D figures. The
        student has to be familiar with creating surface and mesh
        plots.

        \subsubsection{Theme 8: Data Handling}
        The student must be familiar with reading and writing to and
        from \texttt{CSV} files.%, as well as the various file types in Python. 
        The student must be able to store the data in a given
        format. The student must also understand how to access data,
        stored by Python, in LibreOffice and \textit{vice versa}.

        \subsubsection{Theme 9: High Level Programming}
                The student must be able to navigate Python's documentation
                as well as online documentation and be able to find, read and
                understand the documentation of the required functions in order
                to solve specific problems. The student should therefore be
                able to independently increase his/her knowledge of the Python
                programming language.

                The student must then be able use these additional modules and
                functions in Python in order to solve a specific problem.

        \subsubsection{Theme 10: Graphical User Interfaces (GUI's)}
            The student has to be familiar with the Tkinter module for creating
            simple GUI's. The student has to be familiar with creating simple
            form objects (e.g. buttons, input boxes, text boxes, etc.) and be
            able to create these form objects on a GUI. The student also has to
            be familiar with handling user events (keyboard or mouse events)
            and be able to link these events to functions in order to solve
            a given problem through using a GUI.

        \subsubsection{Theme 11: Spreadsheets}
            \paragraph{Formulas and Calculations:}
                The student must be able to solve problems using formulas. The
                student must be able to solve problems using functions that he/she
                are familiar with, as well as use the function wizard for unknown
                functions. The student must be able to name ranges of cells and use
                them in calculations.

                The student must be able to use filters to filter data and perform
                calculations on the filtered dataset. The student must be able to
                perform calculations on data that is split over multiple
                spreadsheets.

            \paragraph{Spreadsheet Detective:}
                The student must be able to use the spreadsheet detective to
                familiarise themselves with the dependencies in an unknown
                spreadsheet. The student must be able to switch between formula
                and value view.

            \paragraph{Plotting and Graphs:}
                The student must be able to create and customise charts. The
                student must be able to find trends in data.

            \paragraph{Linear Programming:}
                The student must be able to solve linear programming problems
                using LibreOffice Calc. The student must be able to solve
                problems with mixed integer, real and boolean variables. The
                student must be able to accommodate multiple constraints.

            \paragraph{Non-Linear Solver:}
                The student must be able to solve non-linear one dimensional
                problems using goal seek in LibreOffice Calc.

            \paragraph{Data Pilot:}
                The student must be able to extract relational data using the Data
                Pilot in LibreOffice Calc.

            \paragraph{What-if Scenarios:}
                The student must be able to perform what-if investigations using
                the scenario function in LibreOffice Calc.

            \paragraph{Visual Formatting:}
                The student must be able to improve the appearance of a worksheet
                by using colours, highlighting, borders and font properties. The
                student must be able to perform conditional formatting of cells.

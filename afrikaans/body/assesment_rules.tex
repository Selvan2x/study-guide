\section{Assesserings Proses}
    Verwys na die eksamen regulasies in die Jaarboek van die Fakulteit
    Ingenieurswese, Bou-Omgewing en IT.

    Om die module deur te kom moet die student:
    \begin{itemize}
        \item 'n Finale punt van 50\% behaal; {\bf en}
        \item 'n Sub-minimum van 40\% vir die semester punt; {\bf en}
        \item 'n Sub-minimum van 40\% vir die finale eksamen
    \end{itemize}

    \subsection{Berekening van die Finale Punt}
        Die finale punt word as volg bereken:
        \begin{itemize}
            \item Semester punt: 50\%
            \item Finale eksamen: 50\% (3-uur eksamen), geslote boek
              agter 'n rekenaar
        \end{itemize}

    \subsection{Berekening van die Semesterpunt}
        Besonderhede rakende die berekening van die semesterpunt word in die
        volgende tabel gelys:
        \begin{table}[!h]
            \begin{center}
             \begin{tabular}{|p{5cm}|c|l|l|}
               \hline
               {\bf Evaluasie Metode} & {\bf Aantal} &
               {\bf Bydrae van elkeen} & {\bf Totaal} \\
               \hline
               Semestertoetse, geslote boek
               agter 'n rekenaar
               & 2 & 40\% & {\bf 80\%} \\ \hline
               Semesterprojek
               & 1 & 10\% & {\bf 10\%} \\ \hline
               Tutoriaalopdragte
               & \multicolumn{2}{|c|}{Al} & {\bf 10\%} \\
               \hline
               \multicolumn{3}{|l|}{{\bf Total}} & {\bf 100\%} \\
               \hline
             \end{tabular}
             \caption{Semesterpunt berekening}
            \end{center}
        \end{table}

    \subsection{Semestertoetse}
        Twee semester toetse sal geskryf word gedurende die semester, in die
        week van ?? ?? to ?? ?? 2017 en ?? tot ?? ?? 2017. Semester
        toetse is 90 minute lank. Die sillabus wat in die toets gedek word sal
        die week voor toetsweek bespreek word. Altwee toetse sal geslote boek
        wees. Die toetse sal in die rekenaar labs geskryf word waar die
        studente toegang sal h\^{e} tot Python en LibreOffice.

        Bykomende semester toets instruksies sal die voorafgaande week tydens
        die lesings gegee word en op \textit{ClickUP} gesit word.

        Die oplossingsblad van die geskeduleerde toetse sal in elektroniese
        formaat op \textit{ClickUP} beskibaar wees na die onderskeie toetse.
        Die student is welkom om gebruik te maak van die besprekingsbord op
        \textit{ClickUP} vir enige verdere vrae oor die toetse.

    \subsection{Tutoriaalpunte} \label{sec:tutoriaal}
        Tutoriaal vrae wat aangedui is vir inhandiging moet op \textit{ClickUP}
        gelaai word voor die inhandiging. \textbf{Neem kennis dat hierdie
        opdragte elektronies gemerk gaan word, waar die student op elke
        vraag onderskeidelik slegs 0\% of 100\% kan verwerf. Dit is
        daarvoor uiters belangrik dat die inhandigingsinstruksie op
        le\^ernaam, objeknaam en funksienaam noukeurig gevolg word.
        Versuiming hiervan sal lei tot 0\% vir die vraag.}

        \textbf{Let Op:} Tutoriaal opdragte moet op 'n individu\"ele basis
        gedoen word. Alle elektroniese opdragte wat ingedien word sal vir
        plagiaat nagegaan word. Enige plagiaat oortredings sal nie toegelaat
        word nie.  Verwys na die Department\"ele Studiegids (sien afdeling
        \ref{sec:department}) verdere inligting oor plagiaat.

    \subsection{App\`{e}lle en navrae oor punte}
        Die punte toegeken vir die tutoriaalopdragte en semestertoetse sal
        beskikbaar gemaak word op \textit{ClickUP}. Indien die student enige
        navrae het oor die toegekende punt moet die student asseblief binne 14
        dae vanaf die punte ontvang is die prosedure onder volg.  Na die
        verloop van 14 dae sal geen veranderings aangebring word nie. Sien die
        departementele studiegids in Afdeling \ref{sec:department} vir verdere
        inligting.

    \subsubsection{App\`{e}l Proses}
        \begin{enumerate}
            \item Laai die memorandum en navraagvorm van \textit{ClickUP} af.
            \item Gaan noukeurig deur jou toetsvraestel en memorandum.
            \item Vul die navraagvorm uit en dui aan wat die navraag behels.
            \item Handig jou vraestel aan die dosent teen die einde van die
                lesing binne die 14 dae periode.
        \end{enumerate}

        \textbf{Geen vraestel sal aanvaar word sonder 'n aangehegde navraagvorm
        nie.}  Die dosent sal wag tot al die vraestelle ontvang is en tot die
        14 dae verby is voor die hersieningsproses aangepak word.

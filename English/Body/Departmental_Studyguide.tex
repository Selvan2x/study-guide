\section{Departmental Study Guide} \label{sec:dep_study_guide}
    This study guide is a crucial part of the general study guide of the
    Department. In the study guide of the Department , information is given
    on the mission and vision of the department , general administration and
    regulations (professionalism and integrity, course related information
    and formal communication, workshop use and safety, plagiarism, class
    representative duties, sick test and sick exam guidelines, vacation work,
    appeal process and adjustment of marks, university regulations, frequently
    asked questions), ECSA outcomes and ECSA exit level outcomes, ECSA
    knowledge areas, CDIO, new curriculum and assessment of cognitive levels.
    It is expected that you are very familiar with the content of the
    Departmental Study Guide. It is available in
    \href{http://web.up.ac.za/sitefiles/file/44/1026/2163/noticeboard/%
        DepartmentalStudyGuide\_Eng\_2014.pdf}{English}
    and
    \href{http://web.up.ac.za/sitefiles/file/44/1026/2163/noticeboard/%
        Departementele\_Studiegids\_Afr\_2014(1).pdf}{Afrikaans}
    on the
    \href{http://web.up.ac.za/default.asp?ipkCategoryID=11426&%
        subid=11426&ipklookid=7}{Department’s website.}

    \noindent
    \textbf{English:} \\
    \url{http://web.up.ac.za/sitefiles/file/44/1026/2163/noticeboard/%
        DepartmentalStudyGuide\_Eng\_2014.pdf} \\
    \textbf{Afrikaans:} \\
    \url{http://web.up.ac.za/sitefiles/file/44/1026/2163/noticeboard/%
        Departementele\_Studiegids\_Afr\_2014(1).pdf} \\~\\

    \noindent
    \textbf{Take note of the \uline{specific instructions} in the above study guide on:}
    \begin{itemize}
        \item \textbf{Safety}
        \item \textbf{Plagiarism}
        \item \textbf{What to do if you were sick (very important)?}
        \item \textbf{Appeal process on the adjustment of marks}
    \end{itemize}
